%% ling.tex - linguistics macros
%% macros for inserting postscript files
%%            numbering examples
%%            recording paragraph indent for use in parboxes
%%            italicizing text w/appropriate following spacing
%%            generating struts

%% command to include non-encapsulated postscript files
%% actually include the files
\newcommand{\PSbox}[3]{\mbox{\rule{0in}{#3}\special{psfile=#1}\hspace{#2}}}
%% show frameboxes in place of the files
%\newcommand{\PSbox}[3]{\framebox{\rule{0in}{#3}\special{psfile=#1}\hspace{#2}}}

% Example(s) Environments; should work at any size, assuming standard list-
%    building parameters (\topsep, \itemsep, etc.) and \baselineskip are
%    properly set.
% No new-lines after example number is printed

\newcounter{examplectr}

% This line is to overcome a bug in cmu-art style: it prints counter
% values to the aux file using \theaux... rather than using \the...
\def\theauxexamplectr{\theexamplectr}

\newcounter{subexamplectr}
\def\theauxsubexamplectr{\thesubexamplectr}

\renewcommand{\theexamplectr}{\arabic{examplectr}}
% This command causes example numbers to appear without following periods

\renewcommand{\thesubexamplectr}{\theexamplectr\alph{subexamplectr}}
% This command gives the number of an example and subexample as e.g. 1a, 2b

\newcommand{\exref}[1]{(\ref{#1})}
% This command puts reference numbers with parentheses
% surrounding them 

% The following are replaced by the more font-sensitive versions farther down.
%
% The environment ``examples'' gives a list of examples, one on each line,
% numbered with a lower case alphabetic character
%\newenvironment{examples}%
%        { \vspace{-\baselineskip}
%          \begin{list}%
%          {\rm \alph{subexamplectr}.}%
%          {\usecounter{subexamplectr}
%          \setlength{\topsep}{-\parskip}
%          \setlength{\itemsep}{-2pt}
%          \setlength{\leftmargin}{0.5in}
%          \setlength{\rightmargin}{0in} } }%
%        { \end{list}}
%
%% The environment ``myexample'' outputs an arabic counter ``examplectr''
%% surrounded by parentheses.
%\newenvironment{myexample}
%        { \vspace{3ex}
%          \noindent
%          \begin{minipage}{\textwidth}    % minipage environment disallows
%                                          % breaks across pages
%
%          \refstepcounter{examplectr}     % step the counter and cause this
%                                          % section to be referenced by the
%                                          % counter ``examplectr''
%          (\arabic{examplectr})}%
%        { \vspace{3ex}
%          \end{minipage}}

% The environment ``examples'' gives a list of examples, one on each line,
% numbered with a lower case alphabetic character.
%
%   NOTE:  eventually should adjust margin lengths to be font-relative
%
\newenvironment{examples}%
        { \vspace{-\baselineskip}
          \begin{list}%
          {\rm \alph{subexamplectr}.}%
          {\usecounter{subexamplectr}
          \setlength{\topsep}{-\parskip}
          \setlength{\itemsep}{-\parsep}
          \setlength{\leftmargin}{0.5in}
          \setlength{\rightmargin}{0in} } }%
        { \end{list}}

% The environment ``myexample'' outputs an arabic counter ``examplectr''
% surrounded by parentheses.
%
%   NOTE:  if two of these occur in sequence in the input file, they should
%          be separated by a blank line, and a \vspace{-.8\baselineskip}
%          command
%
\newlength{\exampwidth}
%\newlength{\exampskip}

\newenvironment{myexample}
        { %\setlength{\exampskip}{0pt}
          %\addtolength{\exampskip}{\baselinestretch\baselineskip}
          %\vspace{.5\exampskip}
          \vspace{.8\baselineskip}
          \setlength{\exampwidth}{\textwidth}
          \addtolength{\exampwidth}{-\parindent}
          \begin{minipage}{\exampwidth}%
          \refstepcounter{examplectr}     % step the counter and cause this
                                          % section to be referenced by the
                                          % counter ``examplectr''
          (\arabic{examplectr})}%
        { %\vspace{.5\exampskip}
          \vspace{.8\baselineskip}
          \end{minipage}}

% record paragraph indent for use in parboxes (which set it to zero)
\newlength{\parboxind}
\setlength{\parboxind}{\parindent}

%% commands to display thematic grids

%% with underlining
%% \newcommand{\unagrid}[1]{$<$#1$>$}                     %single int arg
%% \newcommand{\unegrid}[1]{$<$\underline{#1}$>$}         %single ext arg
%% \newcommand{\trngrid}[2]{$<$\underline{#1}, #2$>$}     %ext and int args
%% \newcommand{\datgrid}[3]{$<$\underline{#1}, #2, #3$>$} %three args

%% without underlining
%% \newcommand{\unegrid}[1]{$<$#1$>$}         %single ext arg
%% \newcommand{\trngrid}[2]{$<$#1, #2$>$}     %ext and int args
%% \newcommand{\datgrid}[3]{$<$#1, #2, #3$>$} %three args


%% command to generate a strut
\newcommand{\spacer}[2]{\rule[#1ex]{0em}{#2ex}}

%% commands to italicize text
%% \newcommand{\tx}[1]{{\it #1\/}}  %% for text followed by roman type
%% \newcommand{\tz}[1]{{\it #1}}    %% for text followed by . or ,

%% (better) way to handle italic text - this takes care of the correction 
%%             handled by hand in \tx and \tz above
\newcommand{\ext}[1]{\textit{#1}} %\ext{arg} for text examples

%% EOF %%
