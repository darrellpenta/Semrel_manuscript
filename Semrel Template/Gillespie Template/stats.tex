%% stats.tex - statistics-presentation macros
%% macros for presenting statistical tests

%% basic F=value w/2 df terms
\newcommand{\Fval}[3]{\ensuremath{F(#1,#2)=#3}}

%% basic F w/subscript (optarg1=sign, default==; arg2=subscript;
%%                       arg3=df1; arg4=df2; arg5=val)
%%\newcommand{\Fsub}[5][=]{\mbox{\ensuremath{F_{#2}(#3,#4) #1 #5}}}
\newcommand{\Fsub}[5][=]{\ensuremath{F_{#2}(#3,#4)} \ensuremath{#1 #5}}

%% basic MSe (optarg1=sign, default==; arg2=val)
\newcommand{\MSe}[2][=]{\ensuremath{M\!S_{e}} \ensuremath{#1 #2}}

%% MSe < val or MSe > val
%\newcommand{\MSes}[2][=]{\mbox{\ensuremath{M\!S_{e} #1 #2}}}

%% basic mean (optarg1=sign, default==; arg2=val)
\newcommand{\mean}[2][=]{\mbox{\ensuremath{M #1 #2}}}

%% basic SD (optarg1=sign, default==; arg2=val)
\newcommand{\SD}[2][=]{\mbox{\ensuremath{S\!D #1 #2}}}

%% F1 and F2 (optarg1=df num (default=1), arg2=df denom, arg3=val,
%%             arg4=MSe)
%\newcommand{\Fs}[4][1]{\Fsub{1}{#1}{#2}{#3}}
%\newcommand{\Fi}[4][1]{\Fsub{2}{#1}{#2}{#3}}
\newcommand{\Fs}[4][1]{\Fsub{1}{#1}{#2}{#3}, \MSe{#4}}
\newcommand{\Fi}[4][1]{\Fsub{2}{#1}{#2}{#3}, \MSe{#4}}
\newcommand{\Fso}[3][1]{\Fsub{1}{#1}{#2}{#3}}
\newcommand{\Fio}[3][1]{\Fsub{2}{#1}{#2}{#3}}

%% F1 or F2 < val (optarg1=val, default=1; arg2=subscript)
\newcommand{\Fsubweak}[2][1]{\mbox{\ensuremath{F_{#2}<#1}}}

%% F1 or F2 > val (optarg1=val, default=1; arg2=subscript)
\newcommand{\Fsubgood}[2][1]{\mbox{\ensuremath{F_{#2}>#1}}}

%% F1 or F2 w/df and sign (optarg1=df num, default=1; arg2=df denom;
%%                          arg3=sign; arg4=val)
\newcommand{\Fsmult}[4][1]{\Fsub[#3]{1}{#1}{#2}{#4}}
\newcommand{\Fimult}[4][1]{\Fsub[#3]{2}{#1}{#2}{#4}}

%% F < val (optarg=val, default=1)
\newcommand{\Fweak}[1][1]{\mbox{\ensuremath{F < #1}}}

%% Fs < val (optarg=val, default=1)
\newcommand{\Fsweak}[1][1]{\mbox{\ensuremath{F}s \ensuremath{< #1}}}

%% Fs > val
\newcommand{\Fsgood}[1]{\mbox{\ensuremath{F}s \ensuremath{> #1}}}

%% basic t (optarg=sign, default==; arg2=df; arg3=val)
\newcommand{\tval}[3][=]{\ensuremath{t(#2)} \ensuremath{ #1 #3}}

%% basic t=val w/subscript (optarg=sign, default==; arg2=subscript;
%%                           arg3=df; arg4=val)
\newcommand{\tsub}[4][=]{\ensuremath{t_{#2}(#3)} \ensuremath{#1 #4}}

%% t1 and t2 (optarg=sign, default==; arg2=df, arg3=val)
\newcommand{\ts}[3][=]{\tsub[#1]{1}{#2}{#3}}
\newcommand{\ti}[3][=]{\tsub[#1]{2}{#2}{#3}}

%% ts < val or ts > val
\newcommand{\tmult}[2]{\mbox{\ensuremath{t}s \ensuremath{#1 #2}}}

%% basic z=val
\newcommand{\zval}[1]{\ensuremath{z=#1}}

%% zs < val (optarg=val, default=1)
\newcommand{\zsweak}[1][1]{\mbox{\ensuremath{z}s \ensuremath{< #1}}}

%% basic r=val
\newcommand{\rval}[1]{\mbox{\ensuremath{r=#1}}}

%% mult Rsq=val
\newcommand{\Rsq}[1]{\mbox{\ensuremath{R^{2}=#1}}}

%% basic pval (opt arg is relation, default = lessthan)
\newcommand{\p}[2][<]{\mbox{\ensuremath{p #1 .#2}}}

%% ps > val (optarg=val, default=10)
\newcommand{\psweak}[1][10]{\mbox{\ensuremath{p}s \ensuremath{> .#1}}}

%% ps < val
\newcommand{\psgood}[1]{\mbox{\ensuremath{p}s \ensuremath{< .#1}}}

%% eta-squared=val
\newcommand{\etasq}[1]{\mbox{\ensuremath{\eta^{2}=#1}}}

%% Sample:
%% This is just a test:  Here's \Fs{27}{222.34}{7262.3}, \p{05}; 
%% \Fi[2]{12}{121}{27719}, \p[=]{02}.
%%
%% And some other tests:  This was one weak result: \Fsweak,
%% \psweak.  And this one was pretty weak too: \Fsweak[2], 
%% \psweak[25].

%% EOF %%
