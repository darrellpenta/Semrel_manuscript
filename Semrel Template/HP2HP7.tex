\documentclass[12pt,titlepage]{article}

\usepackage{mathptmx}  %% Times font for text & math

\usepackage{calc}  %% permits arithmetic expressions in \setlength, etc.
\usepackage{graphicx}  %% adds graphics inclusion routines incl box 
                       %% rotation
%\usepackage{rotating}

%old way of setting margins:
%\usepackage[myheadings]{fullpage}  %% use 1in margins all around

\usepackage[centering,margin={1in,1in}]{geometry}

\usepackage{setspace}  %% double space text
\setstretch{1.8}

%\usepackage{array}   %% advanced tabular stuff
%\usepackage{tabularx} %%more advanced tabular stuff
\usepackage{dcolumn}  %% allows decimal-centered table columns
\usepackage{ctable}

%place footnotes at end
% \usepackage{endnotes}
% \renewcommand{\notesname}{}
% \renewcommand{\enotesize}{\normalsize}
% \renewcommand{\footnote}{\endnote}

%\usepackage{nofigcap}  %% no captions w/figs (for final submission)

%page setup for APA:
\setcounter{secnumdepth}{-1}  %% don't number any sections
\pagestyle{myheadings}        %% use user-defined heading (+ page num)
\markright{{\hfill Hierarchy and Scope of Planning\hspace{2em}}} %% page head

%% ling.tex - linguistics macros
%% macros for inserting postscript files
%%            numbering examples
%%            recording paragraph indent for use in parboxes
%%            italicizing text w/appropriate following spacing
%%            generating struts

%% command to include non-encapsulated postscript files
%% actually include the files
\newcommand{\PSbox}[3]{\mbox{\rule{0in}{#3}\special{psfile=#1}\hspace{#2}}}
%% show frameboxes in place of the files
%\newcommand{\PSbox}[3]{\framebox{\rule{0in}{#3}\special{psfile=#1}\hspace{#2}}}

% Example(s) Environments; should work at any size, assuming standard list-
%    building parameters (\topsep, \itemsep, etc.) and \baselineskip are
%    properly set.
% No new-lines after example number is printed

\newcounter{examplectr}

% This line is to overcome a bug in cmu-art style: it prints counter
% values to the aux file using \theaux... rather than using \the...
\def\theauxexamplectr{\theexamplectr}

\newcounter{subexamplectr}
\def\theauxsubexamplectr{\thesubexamplectr}

\renewcommand{\theexamplectr}{\arabic{examplectr}}
% This command causes example numbers to appear without following periods

\renewcommand{\thesubexamplectr}{\theexamplectr\alph{subexamplectr}}
% This command gives the number of an example and subexample as e.g. 1a, 2b

\newcommand{\exref}[1]{(\ref{#1})}
% This command puts reference numbers with parentheses
% surrounding them 

% The following are replaced by the more font-sensitive versions farther down.
%
% The environment ``examples'' gives a list of examples, one on each line,
% numbered with a lower case alphabetic character
%\newenvironment{examples}%
%        { \vspace{-\baselineskip}
%          \begin{list}%
%          {\rm \alph{subexamplectr}.}%
%          {\usecounter{subexamplectr}
%          \setlength{\topsep}{-\parskip}
%          \setlength{\itemsep}{-2pt}
%          \setlength{\leftmargin}{0.5in}
%          \setlength{\rightmargin}{0in} } }%
%        { \end{list}}
%
%% The environment ``myexample'' outputs an arabic counter ``examplectr''
%% surrounded by parentheses.
%\newenvironment{myexample}
%        { \vspace{3ex}
%          \noindent
%          \begin{minipage}{\textwidth}    % minipage environment disallows
%                                          % breaks across pages
%
%          \refstepcounter{examplectr}     % step the counter and cause this
%                                          % section to be referenced by the
%                                          % counter ``examplectr''
%          (\arabic{examplectr})}%
%        { \vspace{3ex}
%          \end{minipage}}

% The environment ``examples'' gives a list of examples, one on each line,
% numbered with a lower case alphabetic character.
%
%   NOTE:  eventually should adjust margin lengths to be font-relative
%
\newenvironment{examples}%
        { \vspace{-\baselineskip}
          \begin{list}%
          {\rm \alph{subexamplectr}.}%
          {\usecounter{subexamplectr}
          \setlength{\topsep}{-\parskip}
          \setlength{\itemsep}{-\parsep}
          \setlength{\leftmargin}{0.5in}
          \setlength{\rightmargin}{0in} } }%
        { \end{list}}

% The environment ``myexample'' outputs an arabic counter ``examplectr''
% surrounded by parentheses.
%
%   NOTE:  if two of these occur in sequence in the input file, they should
%          be separated by a blank line, and a \vspace{-.8\baselineskip}
%          command
%
\newlength{\exampwidth}
%\newlength{\exampskip}

\newenvironment{myexample}
        { %\setlength{\exampskip}{0pt}
          %\addtolength{\exampskip}{\baselinestretch\baselineskip}
          %\vspace{.5\exampskip}
          \vspace{.8\baselineskip}
          \setlength{\exampwidth}{\textwidth}
          \addtolength{\exampwidth}{-\parindent}
          \begin{minipage}{\exampwidth}%
          \refstepcounter{examplectr}     % step the counter and cause this
                                          % section to be referenced by the
                                          % counter ``examplectr''
          (\arabic{examplectr})}%
        { %\vspace{.5\exampskip}
          \vspace{.8\baselineskip}
          \end{minipage}}

% record paragraph indent for use in parboxes (which set it to zero)
\newlength{\parboxind}
\setlength{\parboxind}{\parindent}

%% commands to display thematic grids

%% with underlining
%% \newcommand{\unagrid}[1]{$<$#1$>$}                     %single int arg
%% \newcommand{\unegrid}[1]{$<$\underline{#1}$>$}         %single ext arg
%% \newcommand{\trngrid}[2]{$<$\underline{#1}, #2$>$}     %ext and int args
%% \newcommand{\datgrid}[3]{$<$\underline{#1}, #2, #3$>$} %three args

%% without underlining
%% \newcommand{\unegrid}[1]{$<$#1$>$}         %single ext arg
%% \newcommand{\trngrid}[2]{$<$#1, #2$>$}     %ext and int args
%% \newcommand{\datgrid}[3]{$<$#1, #2, #3$>$} %three args


%% command to generate a strut
\newcommand{\spacer}[2]{\rule[#1ex]{0em}{#2ex}}

%% commands to italicize text
%% \newcommand{\tx}[1]{{\it #1\/}}  %% for text followed by roman type
%% \newcommand{\tz}[1]{{\it #1}}    %% for text followed by . or ,

%% (better) way to handle italic text - this takes care of the correction 
%%             handled by hand in \tx and \tz above
\newcommand{\ext}[1]{\textit{#1}} %\ext{arg} for text examples

%% EOF %%
  %% load example-numbering macros, etc.
%% stats.tex - statistics-presentation macros
%% macros for presenting statistical tests

%% basic F=value w/2 df terms
\newcommand{\Fval}[3]{\ensuremath{F(#1,#2)=#3}}

%% basic F w/subscript (optarg1=sign, default==; arg2=subscript;
%%                       arg3=df1; arg4=df2; arg5=val)
%%\newcommand{\Fsub}[5][=]{\mbox{\ensuremath{F_{#2}(#3,#4) #1 #5}}}
\newcommand{\Fsub}[5][=]{\ensuremath{F_{#2}(#3,#4)} \ensuremath{#1 #5}}

%% basic MSe (optarg1=sign, default==; arg2=val)
\newcommand{\MSe}[2][=]{\ensuremath{M\!S_{e}} \ensuremath{#1 #2}}

%% MSe < val or MSe > val
%\newcommand{\MSes}[2][=]{\mbox{\ensuremath{M\!S_{e} #1 #2}}}

%% basic mean (optarg1=sign, default==; arg2=val)
\newcommand{\mean}[2][=]{\mbox{\ensuremath{M #1 #2}}}

%% basic SD (optarg1=sign, default==; arg2=val)
\newcommand{\SD}[2][=]{\mbox{\ensuremath{S\!D #1 #2}}}

%% F1 and F2 (optarg1=df num (default=1), arg2=df denom, arg3=val,
%%             arg4=MSe)
%\newcommand{\Fs}[4][1]{\Fsub{1}{#1}{#2}{#3}}
%\newcommand{\Fi}[4][1]{\Fsub{2}{#1}{#2}{#3}}
\newcommand{\Fs}[4][1]{\Fsub{1}{#1}{#2}{#3}, \MSe{#4}}
\newcommand{\Fi}[4][1]{\Fsub{2}{#1}{#2}{#3}, \MSe{#4}}
\newcommand{\Fso}[3][1]{\Fsub{1}{#1}{#2}{#3}}
\newcommand{\Fio}[3][1]{\Fsub{2}{#1}{#2}{#3}}

%% F1 or F2 < val (optarg1=val, default=1; arg2=subscript)
\newcommand{\Fsubweak}[2][1]{\mbox{\ensuremath{F_{#2}<#1}}}

%% F1 or F2 > val (optarg1=val, default=1; arg2=subscript)
\newcommand{\Fsubgood}[2][1]{\mbox{\ensuremath{F_{#2}>#1}}}

%% F1 or F2 w/df and sign (optarg1=df num, default=1; arg2=df denom;
%%                          arg3=sign; arg4=val)
\newcommand{\Fsmult}[4][1]{\Fsub[#3]{1}{#1}{#2}{#4}}
\newcommand{\Fimult}[4][1]{\Fsub[#3]{2}{#1}{#2}{#4}}

%% F < val (optarg=val, default=1)
\newcommand{\Fweak}[1][1]{\mbox{\ensuremath{F < #1}}}

%% Fs < val (optarg=val, default=1)
\newcommand{\Fsweak}[1][1]{\mbox{\ensuremath{F}s \ensuremath{< #1}}}

%% Fs > val
\newcommand{\Fsgood}[1]{\mbox{\ensuremath{F}s \ensuremath{> #1}}}

%% basic t (optarg=sign, default==; arg2=df; arg3=val)
\newcommand{\tval}[3][=]{\ensuremath{t(#2)} \ensuremath{ #1 #3}}

%% basic t=val w/subscript (optarg=sign, default==; arg2=subscript;
%%                           arg3=df; arg4=val)
\newcommand{\tsub}[4][=]{\ensuremath{t_{#2}(#3)} \ensuremath{#1 #4}}

%% t1 and t2 (optarg=sign, default==; arg2=df, arg3=val)
\newcommand{\ts}[3][=]{\tsub[#1]{1}{#2}{#3}}
\newcommand{\ti}[3][=]{\tsub[#1]{2}{#2}{#3}}

%% ts < val or ts > val
\newcommand{\tmult}[2]{\mbox{\ensuremath{t}s \ensuremath{#1 #2}}}

%% basic z=val
\newcommand{\zval}[1]{\ensuremath{z=#1}}

%% zs < val (optarg=val, default=1)
\newcommand{\zsweak}[1][1]{\mbox{\ensuremath{z}s \ensuremath{< #1}}}

%% basic r=val
\newcommand{\rval}[1]{\mbox{\ensuremath{r=#1}}}

%% mult Rsq=val
\newcommand{\Rsq}[1]{\mbox{\ensuremath{R^{2}=#1}}}

%% basic pval (opt arg is relation, default = lessthan)
\newcommand{\p}[2][<]{\mbox{\ensuremath{p #1 .#2}}}

%% ps > val (optarg=val, default=10)
\newcommand{\psweak}[1][10]{\mbox{\ensuremath{p}s \ensuremath{> .#1}}}

%% ps < val
\newcommand{\psgood}[1]{\mbox{\ensuremath{p}s \ensuremath{< .#1}}}

%% eta-squared=val
\newcommand{\etasq}[1]{\mbox{\ensuremath{\eta^{2}=#1}}}

%% Sample:
%% This is just a test:  Here's \Fs{27}{222.34}{7262.3}, \p{05}; 
%% \Fi[2]{12}{121}{27719}, \p[=]{02}.
%%
%% And some other tests:  This was one weak result: \Fsweak,
%% \psweak.  And this one was pretty weak too: \Fsweak[2], 
%% \psweak[25].

%% EOF %%
  %% load macros for presenting stats
%\input{epsf.tex}  %% load macros for handling encapsulated postscript

%some declarations:

\newcommand{\bft}{\textbf}  %% \bft{} gives boldface
\newcommand{\itt}{\textit}  %% \itt{} gives italics

%\usepackage{soul}  %% improved underlining (can break along lines;
                    %%   fixed depth regardless of specific text)
                    %% defines new underlining macro (\ul{text})
%\setuldepth{}

%\newcommand{\phgrp}{\ul}  %% use underlining to mark phrase groups
\newcommand{\phgrp}[1]{[#1]}  %% use bracketing to mark phrase groups

\newlength{\maxcolwidth}  %% stores col header max width
\newlength{\Attwidth}  %% stores %N1 Attachment header width

%\newcolumntype{d}[1]{D{.}{.}{#1}}  %% allows col type d which aligns cols
                                   %% on dec pt; arg is digits on 
                                   %% either side of dec pt
\newcolumntype{S}{D{.}{~}{-1}}  %% allows col type S which aligns cols
                                %% on dec pt & converts dec pt to space
\newcolumntype{T}[1]{D{!}{~}{#1}}  %% allows col type T which aligns cols
                                   %% on !, converts ! to space; arg is
                                   %% dec places around !

%\newsavebox{\savetable}      %% used to store tables so we can get width
%\newlength{\savetablewidth}  %% stores table width so parboxes are OK

%\newcommand{\possep}{\hspace{6\tabcolsep}}  %% extra horiz sep in tables

\newcommand{\insertnote}[1]{%
\vspace{\baselineskip}
\noindent
\begin{minipage}{\textwidth}
\begin{center}
========================= \\
Insert #1 About Here \\
========================= \\
\end{center}
\vspace{.8\baselineskip}
\end{minipage}}

%\newcommand{\framefig}[1]{\framebox{#1}}
\newcommand{\framefig}[1]{#1}

%scale factor for figs that have to be scaled identically (e.g., trees)
\newcommand{\figscaleset}{.45}

\newcommand{\TODO}[1]{\textbf{*** #1 ***}}  %% boldface stuff to do
%\newcommand{\TODO}[1]{}  %% this prevents todo stuff from appearing

\newcommand{\NOTE}[1]{\textbf{*** #1 ***}}  %% boldface notes to me
%\newcommand{\NOTE}[1]{}  %% this version prevents notes from appearing

\newcommand{\GETVAL}{\textbf{*?*}}  %% boldface missing vals

\newcommand{\IGNORE}[1]{} %% text which should not appear in output

\newcommand{\showp}[1]{\IGNORE{#1}} %% text which should not appear in output

%experiment and appx numbering (\@ handles inter-sentence spacing
%when the number or letter appears at the end of a sentence)
%
% \newcommand{\vplbadqs}{1\@}
% \newcommand{\vplnoprev}{2\@}
% \newcommand{\vplgoodqs}{3\@}
% \newcommand{\vpladv}{4\@}
% \newcommand{\thestim}{A\@}
% \newcommand{\theadvstim}{B\@}
% \newcommand{\therawtime}{C\@}

% hyphenation help
\hyphenation{pros-o-dy}

\begin{document}

%\maketitle  %we'll do it ourselves (below)

\begin{titlepage}
\thispagestyle{myheadings}  %% include page head on title page
\begin{center}

\vspace*{\fill}  % equiv of \vfill, but we need the * at bop

% title:
{\Large Hierarchy and Scope of Planning in Subject-Verb Agreement 
Production}

\vfill

% authors & affiliations:
{\large Maureen Gillespie and Neal J. Pearlmutter

Northeastern University

\vfill

% date of LaTeXing
%\today
}
\end{center}

\vfill

\begin{flushleft}
%Send correspondence to: \\

Mailing address: \\
Psychology Dept., 125 NI \\
Northeastern University \\
Boston, MA 02115 \\[\baselineskip]

E-mail: gillespie.m@neu.edu (Gillespie), pearlmutter@neu.edu (Pearlmutter) \\
Phone: (617) 373-3798 (Gillespie), (617) 373-3040 (Pearlmutter) \\
Fax: (617) 373-8714 \\[\baselineskip]

Running head: HIERARCHY AND SCOPE OF PLANNING IN AGREEMENT PRODUCTION
\\[\baselineskip]

%\today\ DRAFT, comments welcome.
In press, \itt{Cognition}; comments welcome. \\
%Please do not cite or quote without permission.
\end{flushleft}

\end{titlepage}

\begin{abstract}
\thispagestyle{myheadings}
\setcounter{page}{2}  %% set page num to 2

Two subject-verb agreement error elicitation studies tested the
hierarchical feature-passing account of agreement computation in production
and three timing-based alternatives: linear distance to the head noun,
semantic integration, and a combined effect of both (a scope of planning
account).  In Experiment~1, participants completed subject noun phrase (NP)
stimuli consisting of a head NP followed by two prepositional phrase (PP)
modifiers, where the first PP modified the first NP, and the second PP
modified one of the two preceding NPs.  Semantic integration between the
head noun and the local noun within each PP was held constant across
structures.  The mismatch error pattern showed an effect of linear distance
to the head noun and no influence of hierarchical distance.  In
Experiment~2, participants completed NP PP PP stimuli in which both PPs
modified the head noun, and both the order of the two PPs and the local
nouns' degree of semantic integration with the head noun were
varied.  The pattern of mismatch errors reflected a combination of
semantic integration and linear distance to the head noun.  These studies
indicate that agreement processes are strongly constrained by
grammatical-level scope of planning, with local nouns planned closer to the
head having a greater chance of interfering with agreement computation.

\vspace{1in}

\noindent Keywords: sentence production; number agreement; syntactic
planning; semantic integration; hierarchical feature-passing; scope of
planning; speech errors; subject-verb agreement

\end{abstract}

\section{ } % Introduction
\setcounter{page}{3}  %% set page num to 3

The study of language production is concerned with how speakers translate
non-verbal thoughts into meaningful grammatical utterances.  While this is
a fairly effortless task that requires little conscious consideration on
behalf of the speaker, the nature of the processes that underlie this task
are complex.  Most language production models (e.g., Bock \&
Levelt, 1994) separate the production planning process into three main
levels: the message level, which represents the speaker's intended meaning;
the grammatical encoding level, which translates the meaning into a
sequence of words; and the phonological encoding level, which translates
the sequence of words into the articulatory plan required to produce the
utterance.  The current work focuses on the grammatical encoding process
and specifically on syntactic planning, which is responsible for creating a
syntactic structure encoding word order, hierarchical syntactic relations,
and inflections.

Inflectional processes in particular have been investigated in a variety of
studies, typically by examining the conditions under which subject-verb
agreement errors can be elicited, as a way of gaining insight into
syntactic planning (e.g., Bock \& Cutting, 1992; Bock \& Miller, 1991;
Franck, Lassi, Frauenfelder, \& Rizzi, 2006; Franck, Vigliocco, \& Nicol,
2002; Hartsuiker, Ant\'{o}n-M\'{e}ndez, \& van Zee, 2001; Solomon \&
Pearlmutter, 2004b; Vigliocco \& Nicol, 1994, 1998).  Bock and Miller
(1991) conducted the first study that elicited subject-verb agreement
errors in a laboratory setting.  They used sentence preambles that were
composed of a head noun followed by a phrase containing a local noun (e.g.,
as in~\exref{BM}).  Subject-verb agreement errors are commonly produced in
sentences containing subject noun phrases with this structure when the head
and local noun mismatch in number\IGNORE{, an effect referred to as
\itt{attraction} or \itt{proximity concord}}.  Experimental items in Bock
and Miller's study manipulated the number marking of the head and local
nouns to form four number conditions.  Conditions in which the head noun
(\ext{key}) and local noun (\ext{cabinet}) had different number markings
(\exref{BMSP}, containing the singular-plural (SP) sequence,
and~\exref{BMPS}, containing the plural-singular (PS) sequence) were
considered the mismatch conditions, while conditions in which the head noun
and local noun had the same number marking (\ref{BMSS},~\ref{BMPP}) were
considered the match conditions.  Preambles were presented auditorily, and
participants were required to repeat them and then complete them as full
sentences.

\begin{myexample}
\label{BM}
\begin{examples}
    
    \item \label{BMSS} The key to the cabinet \hfill (SS)\hspace{22em}
    
    \item \label{BMSP} The key to the cabinets \hfill (SP)\hspace{22em}
    
    \item \label{BMPS} The keys to the cabinet \hfill (PS)\hspace{22em}

    \item \label{BMPP} The keys to the cabinets \hfill (PS)\hspace{22em}
    
\end{examples}
\end{myexample}

Nearly all agreement errors\IGNORE{ ($>90\%$)} occurred in the mismatch
conditions (\ref{BMSP},~\ref{BMPS}).  Within these conditions, agreement
errors were more common when the head noun was singular and the local noun
was plural~\exref{BMSP} than when the head noun was plural and the local
noun was singular~\exref{BMPS}.  This error pattern is referred to as the
mismatch effect and has been replicated in essentially all studies
examining subject-verb agreement (e.g., Bock \& Cutting, 1992; Bock \&
Eberhard, 1993; Bock, Eberhard, Cutting, Meyer, \& Schriefers, 2001; Bock
\& Miller, 1991; Bock, Nicol, \& Cutting, 1999; Eberhard, 1999; Franck et
al., 2006; Hartsuiker et al., 2001; Negro, Chanquoy, Fayol, \&
Louis-Sidney, 2005).  The interference of plural local nouns, and relative
lack of interference of singular local nouns, on subject-verb agreement
provides support for the hypothesis that plural noun forms are marked with
a plural feature, while singular nouns are unmarked (Berent, Pinker,
Tzelgov, Bibi, \& Goldfarb, 2005; Bock \& Eberhard, 1993; Bock \& Miller,
1991; Eberhard, 1997; Eberhard, Cutting, \& Bock, 2005; Vigliocco \& Nicol,
1994, 1998).  While this latter pattern, the plural markedness effect, does
not provide evidence for a specific mechanism for agreement effects, it
does show that mismatch effects are not simply a result of agreement with
the nearest noun and that a more complex mechanism is involved.

Most production research assumes that agreement is implemented through
hierarchical feature-passing (Eberhard et al., 2005; Franck et al., 2002;
Hartsuiker et al., 2001; Vigliocco \& Hartsuiker, 2002; Vigliocco \& Nicol,
1998).  According to this view, agreement is computed once the syntactic
tree structure of a sentence is formed, with number features being passed
up through the subject NP to the verb phrase.  Mismatch effects occur when
a plural feature is inadvertently passed too far up the tree, overwriting
the number from the head noun with the number from a local noun.  Franck et
al.\ (2002) provide the most direct test of the hierarchical
feature-passing hypothesis in an error elicitation experiment using subject
NP preambles containing two PP modifiers, as
in~\exref{Franck-ex}.\IGNORE{\footnote{We focus on singular head versions;
however, Franck et al.\ (2002) did manipulate head noun
plurality.}\NOTE{think it's OK not to point this out}} Their stimuli had a
descending hierarchical structure in which each PP modified the immediately
preceding noun, and the local nouns (\ext{flight} and \ext{canyon}
in~\exref{Franck-ex}) varied in number.  Figure~\ref{descstruc} shows the
syntactic structure as well as the path along which an errant feature from
N2 or N3 would have to pass.

%\insertnote{Figure~\ref{descstruc}}

\begin{figure}[tb]

%    \makebox[\textwidth][c]{\framefig{\includegraphics[width=.75\textwidth]{descstruc}}}
    \makebox[\textwidth][c]{\framefig{\includegraphics[scale=\figscaleset]{descstruc}}}

\caption{Syntactic path a plural feature must travel to interfere with
agreement in Franck et al.'s (2002) stimuli.  The route for a feature from
N2 is shown with solid arrows; the route for a feature from N3 includes the
route from N2 as well as the dashed arrows, so additional feature-passing
errors would have to occur before N3's plural feature could influence verb
number, predicting fewer subject-verb agreement errors when N3 is plural
compared to when N2 is.}

\label{descstruc}
\end{figure}

\begin{myexample}
\label{Franck-ex}
\begin{examples}
    
    \item \label{FranckSSS} The helicopter for the flight over the canyon
    \hfill (SSS)\hspace{12em}
    
    \item \label{FranckSPS} The helicopter for the flights over the canyon
    \hfill (SPS)\hspace{12em}
    
    \item \label{FranckSSP}The helicopter for the flight over the canyons
    \hfill (SSP)\hspace{12em}

    \item \label{FranckSPP}The helicopter for the flights over the canyons
    \hfill (SPP)\hspace{12em}
    
\end{examples}
\end{myexample}

The hierarchical feature-passing hypothesis predicts a larger mismatch
effect for preambles like~\exref{FranckSPS} than for preambles
like~\exref{FranckSSP}.  Because N2 (\ext{flight(s)}) is hierarchically
closer to the verb than N3 (\ext{canyon(s)}) is, fewer feature-passing
errors would have to occur for N2's plural to interfere with agreement
in~\exref{FranckSPS} than for N3's plural to interfere
in~\exref{FranckSSP}.  Franck et al.\ (2002) found that the N2 mismatch
effect was larger than the N3 mismatch effect in both English and French,
and they thus argued for a hierarchical feature-passing account of
subject-verb agreement over a linear account in which interference
increases with (linear) proximity to the verb.\IGNORE{ \NOTE{there's an
issue in the above arg that's largely ignored (by us and others), which is
whether it's overall distance that matters (M\&M basically assumes this;
Franck et al.\ sometimes suggest this), versus the number of nodes that
would have to incorrectly pass number up/along (so NP might not be such a
node, b/c number should be allowed to pass up outside of it; but PP would
be such a node (Franck et al.'s figs 1\&2 suggest this using bold vs roman)
-- this is on analogy to Romance languages, where the P sometimes needs to
be num-marked from N w/in the PP)\ldots I think we should probably not
discuss this here, but it might be relevant for the GD\ldots}}

Current models of agreement computation also assume mechanisms that are
consistent with a hierarchical feature-passing account (Eberhard et al.,
2005; Vigliocco \& Hartsuiker, 2002).  Eberhard et al.'s Marking and
Morphing model was implemented to account for the findings of a number of
agreement studies.  According to this model, the marking process assigns
number to the subject NP as a whole based on message-level
properties\IGNORE{ (this allows conceptual number of the subject phrase to
have an influence independent of the lexically-specified number of any
nouns within the phrase)}.  Separately, each noun within the subject NP is
also assigned a number specification from its lexical entry, and morphing
then combines the subject NP number value set by marking with the number
values from all the nouns within the subject NP, to yield an overall
specification of number for the subject.  This specification in turn
determines the probability of singular versus plural agreement on the verb.
The morphing process encodes the hierarchical distance assumption: Nouns
situated further from the subject NP node in the syntactic tree are
stipulated to have a weaker influence on the subject NP's number assignment
than nouns closer to the subject NP\@.  The implemented model handles only
structures with a head noun and a single local noun at a constant distance
from the subject NP (within a PP), but the assumption is that a local noun
in a more syntactically distant PP (e.g., N3 in Franck et al., 2002) would
have even less of an influence on the subject NP's number assignment.

The hierarchical feature-passing hypothesis can explain various effects
seen in previous studies (see Franck et al., 2002, for discussion of Bock
\& Cutting's, 1992, clause-packaging hypothesis; Vigliocco \& Nicol, 1998,
for additional support from question production; and Hartsuiker et al.,
2001, for cross-linguistic evidence).  However, it does not take into
account the fact that language production is at least partially sequential
or incremental, and that planning of utterances may be as well.  In
particular, it assumes that a full hierarchical structure for the entire
subject NP is available through which features can pass.  Although such
structure must be computed during the course of producing the subject NP,
it may or may not all be present simultaneously, and the part(s) of the
structure relevant to creating mismatch errors (the PPs containing the
local nouns) might not be present at the point in time when the number of
the subject NP is being computed (for a related suggestion, see Haskell \&
MacDonald, 2005).  This possibility suggests a memory-encoding-based
alternative to the hierarchical account: Interference with agreement
computation might be a function of the extent to which the
potentially-interfering element is active in memory at the time when the
number-marking of the subject NP is being computed.  Thus while some local
nouns might be planned close in time to the head, others may be planned
relatively later and thus have a smaller likelihood of creating
interference (see also Pearlmutter \& Solomon, 2007; Solomon \&
Pearlmutter, 2004b, for more details on such a timing of activation
account).  The current work considers two factors, linear distance back to
the head noun and semantic integration (Solomon \& Pearlmutter, 2004b),
which might be expected to affect the relative time of planning of
different elements and which would make predictions compatible with Franck
et al.'s results.

Thus, for example, while Franck et al.\ (2002) showed that a local noun's
linear proximity to the verb cannot account for mismatch effects, no study
to date has examined the influence of a local noun's linear distance back
to the head noun (though see Nicol, 1995, for a related proposal).
Assuming that the number of the subject NP must be computed and retained in
memory, elements linearly closer to the head should be more likely to
interfere with this encoding process, predicting correctly that N2 mismatch
effects should be larger than N3 mismatch effects for Franck et al.'s
stimuli.

Also possible is that, instead of order, the relative timing of planning of
words due to semantic relationships among them may affect agreement
computation.  Solomon and Pearlmutter (2004b) hypothesized that semantic
integration (i.e., the degree to which elements within a phrase are linked
at the message level) affects the timing of planning of elements within a
phrase, such that elements of more semantically integrated phrases are more
likely to be planned simultaneously.  More integrated cases thus produce
more potential interference and a greater possibility for speech errors,
including mismatch errors in agreement error elicitation (see also
Pearlmutter \& Solomon, 2007, for evidence from exchange errors).  Solomon
and Pearlmutter manipulated local noun plurality in NP PP stimuli and
compared integrated cases (e.g., \ext{The pizza with the yummy topping(s)})
to corresponding unintegrated ones (e.g., \ext{The pizza with the tasty
beverage(s)}).  Across a series of experiments, they found larger mismatch
effects for integrated than for unintegrated conditions, supporting the
hypothesis that increased semantic integration leads to increases in
subject-verb agreement mismatch effects by way of increased interference.
This hypothesis can also account for Franck et al.'s (2002) result, as
Solomon and Pearlmutter (2004a) obtained ratings of semantic integration
for Franck et al.'s Experiment~2 (English) stimuli, which showed semantic
integration to be confounded with syntactic distance: N1 and N2 were
significantly more integrated than N1 and N3, predicting correctly that the
N2 mismatch effect should be larger than the N3 mismatch effect, because N1
and N2 would be more likely to be planned simultaneously than N1 and N3.

An additional possibility is that influences of linear order and semantic
integration combine to determine the timing of planning of elements, in
which case nouns linearly closer to the head would be more likely to
interfere, but the extent of interference would be increased by greater
integration with the head and decreased with reduced integration.  This
account suggests that the scope of planning during grammatical encoding may
have an influence on agreement computation.  Research examining exchange
errors provides clear evidence that multiple elements of an utterance are
active simultaneously during production (Garrett, 1975, 1980), suggesting
that speakers plan parts of their utterances in advance of articulation.
However, there is little agreement about the size of the planning units
(cf.\ Allum \& Wheeldon, 2007; Griffin, 2001; Smith \& Wheeldon, 1999,
2001; Wheeldon \& Lahiri, 2002).  Under a scope of planning account, only
local nouns that are within the scope of planning of the head noun when the
number marking of the subject NP is determined would create mismatch
effects, with both decreased head-local linear distance and increased
head-local semantic integration contributing to the likelihood of the local
noun being within the scope of planning of the head, and thus contributing
to the likelihood of the local noun being active simultaneously with the
head noun and creating interference in encoding number.  This scope account
provides an alternative to the hierarchical feature-passing hypothesis
because in Franck et al.\ (2002), N3 was more likely to have been outside
the scope of planning of the head noun when agreement was computed (it was
linearly farthest from the head and weakly integrated with the head).  This
would predict that N3 would be less likely to produce a mismatch effect (or
would produce a weaker one than N2), as they found.

The two experiments below examine hierarchical feature-passing and the
above three alternatives as influences on agreement computation.
Experiment~1 was a direct test of the hierarchical distance hypothesis
against a linear distance alternative, manipulating hierarchical
distance of N3 within preambles (subject NP stimuli) while controlling
semantic integration.  Experiment~2 manipulated semantic integration and
linear order within preambles while controlling hierarchical distance.

\section[Experiment~1]{\center Experiment~1}

To examine hierarchical distance directly and address the semantic
integration confound in Franck et al.'s (2002) stimuli, Experiment~1
compared two sets of NP PP PP preambles, which controlled semantic
integration while manipulating hierarchical distance and local noun number,
as in~\exref{desc} and~\exref{flat}.  (The code following each preamble
indicates noun number for N1, N2, and N3, with S meaning singular and P
plural.)  Preambles like~\exref{desc} had a descending hierarchical
structure, such that the PP containing N3 (\ext{on the leather strap(s)})
modified N2 (\ext{buckle(s)}), as in Figure~\ref{descstruc} and in Franck
et al.\ (except that the current stimuli also had an adjective or noun
modifier of N2 and N3).  Preambles like~\exref{flat} had a flat structure,
such that both PPs (\ext{to the western suburb(s)}, \ext{with the steel
guardrail(s)}) modified N1 (\ext{highway}), as illustrated in
Figure~\ref{flatstruc}.\footnote{The relatively simple, traditional
(Chomsky, 1965) syntactic structures in Figures~\ref{descstruc}
and~\ref{flatstruc} are sufficient to illustrate the contrast in attachment
height (and thus feature-passing distance) of the second PP\@.  Whether
these structures are correct is unknown, and current syntactic theories
differ on the details of the structures for both cases.  However, we are
not aware of any such differences which would alter the general prediction
that the difference in feature-passing distance between N2 and N3 is larger
for the descending than the flat cases.  See the General Discussion for
details on the specific case in which a theory enforces binary branching
(e.g., Chomsky, 1995) and Solomon and Pearlmutter (2004b) for some
discussion of variations in feature-passing predictions depending on
structural details.} Mean semantic integration of the N1-N2 pair was
matched across structures, as was mean semantic integration of the N1-N3
pair.

%\insertnote{Figure~\ref{flatstruc}}

\begin{figure}[tb]

%    \makebox[\textwidth][c]{\framefig{\includegraphics[width=.75\textwidth]{flatstruc}}}
    \makebox[\textwidth][c]{\framefig{\includegraphics[scale=\figscaleset]{flatstruc}}}

\caption{Syntactic structure for flat stimuli, in which both PPs attach to 
the first NP.}

\label{flatstruc}
\end{figure}

\begin{myexample}
\label{desc}
\begin{examples}
    
    \item \label{desc-sss} The backpack with the plastic buckle on the 
    leather strap \hfill (SSS)\hspace{6em}
    
    \item \label{desc-sps}The backpack with the plastic buckles on the 
    leather strap \hfill (SPS)\hspace{6em}
    
    \item \label{desc-ssp} The backpack with the plastic buckle on the 
    leather straps \hfill (SSP)\hspace{6em}
    
\end{examples}
\end{myexample}
\vspace{-1\baselineskip}

\begin{myexample}
\label{flat}
\begin{examples}
    
    \item \label{flat-sss} The highway to the western suburb with the steel
    guardrail \hfill (SSS)\hspace{6em}
    
    \item \label{flat-sps} The highway to the western suburbs with the 
    steel guardrail \hfill (SPS)\hspace{6em}
    
    \item \label{flat-ssp} The highway to the western suburb with the steel
    guardrails \hfill (SSP)\hspace{6em}
    
\end{examples}
\end{myexample}

If hierarchical distance has an independent effect on agreement
computation, the difference between the N2 and N3 mismatch effects should
be smaller for the flat preambles than the descending preambles, because
the distance N2's plural feature would have to travel to affect agreement
computation is matched across structures, while the distance N3's plural
feature would have to travel is shorter in flat structures than in
descending structures.  On the other hand, if the effects Franck et al.\
(2002) attributed to hierarchical distance were instead due to linear
distance back to the head noun, then both flat and descending structures
should show the pattern Franck et al.\ found for their descending
structures --- a higher mismatch error rate for N2 mismatches than for N3
mismatches --- with no interaction between structure and mismatch position.
Finally, if the effects Franck et al.\ found were due to their stimuli's
semantic integration confound (alone), then no interaction should be
observed in this experiment; and in addition, the N2 and N3 mismatch
effects should differ only to the extent that integration between each of
those nouns and N1 varies.

\subsection{Method}

\paragraph{Participants.} Fifty-four Northeastern University students and
community members participated in the on-line experiment.  In this and
Experiment~2, all participants were native English speakers and received
either course credit or payment (\$10) for their participation; no
participant provided data for more than one part of any experiment.

\paragraph{Materials and design.} Twelve descending stimulus items
like~\exref{desc}, chosen from a candidate set of~29, and twelve flat
stimulus items like~\exref{flat}, chosen from a candidate set of~34, were
used as critical items for the experiment.  All preambles consisted of a
head NP (always \ext{The} and a singular head noun, N1) followed by two
PPs, each of which consisted of a preposition, the determiner \ext{the}, an
adjective or modifier noun, and a local noun (N2 or N3).  In the descending
stimuli, the first PP modified the head NP, and the second PP modified the
second NP (containing N2); in the flat cases, both PPs modified the head
NP\@.  All nouns in the stimuli were inanimate and had regular plural
forms, and each noun's conceptual number matched its grammatical
number.\IGNORE{\TODO{syllable-length matching?}} The full set of stimuli is
shown in Appendix~A.

Because simultaneously controlling semantic integration and
manipulating hierarchical structure limited the total number of
available items, only the three local noun number conditions critical
for examining mismatch effects (SSS, SPS, SSP; as shown
in~\exref{desc} and~\exref{flat}) were included in the critical
stimuli, to maximize power.  Thus N1 in the critical stimuli was
always singular, and the three different versions of each item were
created by varying either N2 or N3 number.

Eighty-eight filler preambles were combined with the critical items.  Eight
consisted of an NP PP PP sequence with a singular head noun and plural N2
and N3, in order to balance the SSS, SPS, and SSP critical items; four of
these fillers had descending structures, and four were flat.  Of the other
80 fillers, 32 consisted of an NP PP PP sequence (with varying local noun
number) but had a plural head noun.  The rest had a variety of structures
varying in head noun number and were similar in length and complexity to
the critical items.  The critical items and fillers were combined to form
three counterbalanced lists, each containing all fillers and exactly one
version of each of the critical items.  Each list was seen by~18 
participants.

\paragraph{Stimulus norming.} The~24 critical stimuli were chosen from
the initial~63 candidate stimuli based on two norming studies
conducted in advance, one for semantic integration and one for
attachment of the second PP\@.  While the critical stimuli ended up
instantiating only three local noun number conditions, all four
possible local noun number combinations were normed (SSS, SPS, SSP,
SPP)\@.  Both norming surveys also included an additional~24 filler
stimuli with the same NP PP PP format; local noun number was varied
between-items in these stimuli (6 items per local noun number
condition), as was attachment of the second PP (12 were designed to be
N1-attached and 12 N2-attached), and they were intended to have a
range of levels of semantic integration between N1, N2, and N3.  

The first norming survey, completed by~117 participants, was used to
ensure that the preambles controlled semantic integration as desired.
The~12 different versions of each of the~63 candidate stimulus items
(4 number conditions $\times$ 3 possible rating pairs (N1-N2, N1-N3,
N2-N3)), along with the~24 fillers, were rated for integration
following the procedure described in Solomon and Pearlmutter (2004b).
Participants rated integration of the two underlined nouns in each
preamble, using a~1 (loosely linked) to~7 (tightly linked) scale.  The
instructions included example phrases (\ext{the ketchup or the
mustard} and \ext{the bracelet made of silver}) and indicated that
although \ext{ketchup} and \ext{mustard} are similar in meaning, they
are not closely related in the particular example phrase, in contrast
to \ext{bracelet} and \ext{silver}, which are closely related in the
example phrase.  The~12 versions of each candidate item for rating
were counterbalanced across~12 rating lists such that exactly one
version of each stimulus item appeared in each list.  The~87 preambles
in each list were presented over~5 printed pages, and the pages of
each list were randomized separately for each participant.  Each
participant rated the stimuli in one list, and~9--10 ratings were thus
obtained for all but one version of one stimulus item (which had 
only~8).

Table~\ref{HP7-dat} shows mean ratings by condition and rating pair for
the~24 critical stimuli used in the on-line experiment.  A 2 (structure)
$\times$ 3 (number) $\times$ 3 (rating pair) ANOVA on these data revealed
main effects of structure\footnote{Throughout all experiments, patterns
reported as reliable were significant at or beyond the .05 level unless
otherwise noted.} (\Fval{1}{22}{7.71}, \MSe{2.33}\showp{, \p{05}}) and
rating pair (\Fval{2}{44}{31.39}, \MSe{1.59}\showp{, \p{001}}), and an
interaction of the two (\Fval{2}{44}{8.98}, \MSe{1.59}\showp{, \p{01}}),
with no effect nor interactions involving number (all \Fsweak[1.6],
\psweak[20]).  The main effect of structure and the structure interaction
with rating pair resulted only from different N2-N3 ratings for descending
and flat structures (\Fval{1}{22}{31.32}, \MSe{0.49}\showp{, \p{001}}), as
neither the N1-N2 nor the N1-N3 ratings differed across structures
(\Fsweak); and an additional structure $\times$ number $\times$ rating pair
ANOVA, leaving out the N2-N3 rating pair, showed only a main effect of pair
(N1-N3 more integrated than N1-N2; \Fval{1}{22}{16.17}, \MSe{2.28}\showp{,
\p{01}}).  The critical rating pair by structure interaction in this ANOVA
did not approach significance (\Fweak), indicating that N1-N2 versus N1-N3
integration was matched across structures, as desired.  There were no main
effects of structure or number (\Fsweak) and no interactions involving
these factors (\Fsweak[1.3], \psweak[30]).  These results indicate that
semantic integration was controlled for the critical comparison between the
N2 and N3 mismatch effects within each structure.  The large difference in
N2-N3 integration across structures was also expected given the difference
in attachment of the PP containing N3.

%\insertnote{Table~\ref{HP7-dat}}

{
%\small  %%use if we need to set the table in 11pt instead of 12pt
\settowidth{\maxcolwidth}{Uninflected}
%\settowidth{\maxcolwidth}{999 (99)} %% use this version with Uninfl 
                                     %% col head
\settowidth{\Attwidth}{Attachment}

\setstretch{1}  %% single-spacing for the table

\ctable[
        caption = {\raggedright Experiment~1 Mean Semantic Integration Ratings, 
                   Mean Attachment Preferences, and Response Counts by 
                   Condition},
        cap = {},
        captionskip=2ex,
        label = HP7-dat,
        sideways, 
        star
       ]{l l T{4.6} T{4.6} T{4.6} T{4.6} S@{}S@{}S@{}T{5.0}}{
        \tnote[]{\normalsize\spacer{0}{4}\textit{Note}.
                 Conditions are indicated by noun number (N1, N2, N3),
                 with S = singular and P = plural.  Semantic
                 integration rating scale was 1 (loosely linked) to 7
                 (tightly linked).  For semantic integration ratings
                 and attachment, means are computed by-items and
                 standard deviations are in parentheses.  For response
                 counts, dysfluency counts are in parentheses.  Misc =
                 Miscellaneous.}}{%
        \hline
            \spacer{-2}{6} & & \multicolumn{3}{c}{Semantic Integration Rating} & 
            & \multicolumn{4}{c}{Response Count} \\ \cline{3-5} \cline{7-10}
            \multicolumn{2}{l}{\spacer{-2}{6}Condition} & 
            \multicolumn{1}{c}{N1--N2} &  \multicolumn{1}{c}{N1--N3} & 
             \multicolumn{1}{c}{N2--N3} & 
             \multicolumn{1}{c}{\raisebox{0ex}[0ex][0ex]{\parbox[b]{\Attwidth}{\center \%N1 Attachment}}} & 
             \multicolumn{1}{c@{}}{\makebox[\maxcolwidth][c]{Error}} &
             \multicolumn{1}{c@{}}{\makebox[\maxcolwidth][c]{Correct}} &
             \multicolumn{1}{c@{}}{\makebox[\maxcolwidth][c]{Uninflected}} &
             \multicolumn{1}{c}{\makebox[\maxcolwidth][c]{Misc}} \\ \hline
             \multicolumn{4}{l}{\spacer{-2}{6}Descending} \\
             SSS & & 4.13!(.94) & 5.23!(.49) & 4.46!(.83) & 4.7!(7.01) & 1.(0) & 145.(19) & 37.(5) & 33 \\
             SPS & & 4.34!(.79) & 5.08!(.47) & 4.22!(.72) & 5.6!(5.57) & 15.(2) & 129.(13) & 29.(5) & 43 \\
             SSP & & 4.02!(.77) & 5.21!(.45) & 4.20!(.53) & 4.1!(5.52) & 2.(0) & 124.(12) & 31.(5) & 59 \\
             \spacer{-2}{6}\textit{M} & & 4.16!(.83) & 5.17!(.46) & 4.29!(.70) & 4.8!(5.93) \\ \hline
             \multicolumn{4}{l}{\spacer{-2}{6}Flat} \\
             SSS & & 4.21!(1.38) & 5.24!(.98) & 2.81!(.94) & 93.6!(5.68) & 1.(0) & 148.(20) & 38.(4) & 29 \\
             SPS & & 4.11!(1.36) & 4.96!(.99) & 2.62!(.91) & 91.9!(6.68) & 19.(2) & 114.(10) & 31.(5) & 52 \\
             SSP & & 3.97!(1.23) & 5.13!(.97) & 2.64!(.75) & 94.4!(5.57) & 5.(2) & 128.(10) & 33.(2) & 49 \\
             \spacer{-2}{6}\textit{M} & & 4.10!(1.29) & 5.11!(.96) &  2.69!(.85) & 93.3!(5.92) \\ \hline
             \multicolumn{2}{l}{\spacer{-2}{6}Grand Mean} & 4.13!(1.08) & 5.14!(.75) & 3.49!(1.12) & 49.0!(45.0) \\ \hline
         }
}

The second norming survey, completed by~59 participants\IGNORE{actually 60
Ss; one dropped for ``apparently not understanding the instructions''; may
want to mention, but probably not critical}, ensured that hierarchical
distance was manipulated as desired by measuring attachment of the second
PP to N1 versus N2.  Each preamble was presented followed by a question,
which was always \ext{What is} plus a PP from the preamble.  The question
always asked about the final PP for the candidate stimuli (e.g., \ext{What
is on the leather straps?} for~\exref{desc-ssp}, \ext{What is with the
steel guardrail?} for~\exref{flat-sss}); for the fillers, the question
always asked about the first PP (containing N2), to ensure that
participants paid attention to the full text of each item and not just the
last PP\@.  Participants were instructed to write down the word from the
preamble that best answered the question.  The~4 different local noun
number versions of each candidate preamble were counterbalanced across~4
lists such that exactly one version of each item appeared in each list.
The~63 candidate preambles and~24 fillers in each list were presented
over~8 printed pages, and the pages were randomized separately for each
participant.  Each participant completed a single list, resulting in 14--15
usable responses for each version of each candidate stimulus item.

The responses were coded for whether they referred to N1 or to N2 (unclear
or uninterpretable responses, 1\% of the total\IGNORE{51 ratings out of
5133 (includes all stim and all 59 Ss)}, were excluded), and
Table~\ref{HP7-dat} shows mean preference for N1 attachment by condition
for the~24 critical stimulus items.  A 2 (structure) $\times$ 3 (number)
ANOVA on the \%N1 attachment data revealed a stronger N1 attachment
preference for flat stimuli than for descending stimuli
(\Fval{1}{22}{2201}, \MSe{64.09}\showp{, \p{001}}), as desired.  No main
effect of local noun number and no interaction were present (\Fsweak[1.2],
\psweak[30]).  In addition to differing in direction as desired, the flat
and descending attachment preferences were equally strong in their
respective directions: In the flat stimuli, the second PP attached to N1
over 90\% of the time, while the second PP in the descending stimuli
attached to N1 less than 10\% of the time; the strength of these
preferences relative to 50\% did not differ (\Fweak[1.1], \p[>]{30}).

\paragraph{Apparatus and procedure.} Each participant was run individually
in the on-line experiment using the visual-fragment completion paradigm
(e.g., Bock \& Eberhard, 1993; Solomon \& Pearlmutter, 2004b; Vigliocco,
Butterworth, \& Garrett, 1996; Vigliocco \& Nicol, 1998; cf.\ Haskell \&
MacDonald, 2003).  Participants were instructed to begin reading each
visually-presented preamble aloud as soon as it appeared and add an ending
that formed a complete sentence.  Participants were not instructed as to
how they should formulate a completion, only that they should form a
complete sentence for each preamble.

On each trial, a fixation cross appeared at the left edge of the computer
screen for 1000~ms, followed by the preamble.  As soon as the preamble
appeared, the participant began speaking it aloud, continuing it as a
complete sentence.  Each preamble was presented for the longer of 1000~ms
or 50~ms/character.  After the preamble disappeared, the screen was blank
for 2000~ms, followed by a prompt to begin the next trial.  A PC running
the MicroExperimental Laboratory software package (Schneider, 1988)
presented the preambles, and participants' responses were recorded
uncompressed onto CD-R for analysis, using a Shure SM58 microphone
connected to a Mackie 1202-VLZ Pro mixer/preamp and an Alesis Masterlink
ML-9600 (OS v2.20) CD recorder.  Five practice items preceded the 112
trials.  If at any point the participant's speech rate slowed, the
experimenter encouraged the participant to speak more quickly.

\paragraph{Scoring.} All completions were transcribed and assigned to one
of four coding categories: (1) correct, if the participant repeated the
preamble correctly, only once, produced an inflected verb immediately after
the preamble, and used a verb form that was correctly marked for number;
(2) error, if all the criteria for a correct response were met, but the
verb form failed to agree in number with the subject; (3) uninflected, if
all the criteria for a correct response were met, but the verb was
uninflected; and (4) miscellaneous, if the participant made an error
repeating the preamble, if a verb did not immediately follow the preamble,
or if the response did not fall into any of the other categories.  Trials
in which a participant made no response were excluded from all analyses.
If the participant produced a dysfluency (e.g., pauses, coughs) during or
immediately after producing the preamble and went on to produce a correct,
error, or uninflected response, the scoring category and the dysfluency
were recorded.  On miscellaneous trials, dysfluencies were not scored.

\subsection{Results}

Across all critical trials, there were~788 correctly-inflected
responses,~43 agreement errors,~199 uninflected responses,~265
miscellaneous cases, and~1 trial with no response.  Table~\ref{HP7-dat}
shows the counts for each analyzed response type by condition.  Separate
analyses were performed for error rates (the percentage of error responses
out of error plus correct responses), the number of uninflected responses,
and the number of miscellaneous responses.  All analyses except those
involving miscellaneous counts included dysfluency cases, and unless
otherwise noted, the statistical patterns were identical if dysfluency
cases were excluded.  We also computed supplemental analyses on error
counts and on arcsine-transformed proportions of errors (Cohen \& Cohen,
1983); these are detailed only when they produced significance patterns
different from those for the main error rate analyses.  For each of these
measures, we computed mismatch effects by subtracting from each condition
with a plural local noun (SPS, SSP) the corresponding (structure-matched)
singular baseline (SSS)\@.  The relevant mismatch effects for each of the
measures above were then analyzed in corresponding 2 (structure) $\times$ 2
(plural position) ANOVAs, one with participants ($F_{1}$) and one with
items ($F_{2}$; Clark, 1973) as the random factor.\IGNORE{\NOTE{covered in
fn in Method section} In all analyses, effects reported as reliable were
significant at or beyond the .05 level\IGNORE{ unless otherwise
noted}.}\IGNORE{

old stuff describing overall ANOVAs:

For each of these measures, we performed two overall 2 (structure) $\times$
3 (local noun number) ANOVAs, one with participants ($F_{1}$) and one with
items ($F_{2}$; Clark, 1973) as the random factor.  However, all
theoretical predictions specifically concern mismatch effects, so we
calculated these for each mismatch condition (SPS, SSP) by subtracting out
its corresponding (structure-matched) singular baseline (SSS)\@.  The
relevant mismatch effects for each of the measures above were then analyzed
in corresponding 2 (structure) $\times$ 2 (plural position)
ANOVAs.\IGNORE{\NOTE{covered in fn in Method section} In all analyses,
effects reported as reliable were significant at or beyond the .05
level\IGNORE{ unless otherwise noted}.}

}

\paragraph{Agreement errors.}\IGNORE{

NOTHING IN HERE IS UPDATED FOR S#54

old stuff describing overall ANOVAs:

The analyses of overall agreement error rates
revealed only a main effect of local noun number
(\Fs[2]{104}{14.98}{250.56}\showp{, \p{001}};
\Fi[2]{44}{18.84}{58.13}\showp{, \p{001}}), which arose from a higher error
rate in the SPS condition than in the SSP (\Fs{52}{17.34}{154.73}\showp{,
\p{001}}; \Fi{23}{16.43}{87.94}\showp{, \p{001}}) or the SSS condition
(\Fs{52}{17.01}{172.81}\showp{, \p{001}}; \Fi{23}{25.33}{71.81}\showp{,
\p{001}}).  The SSP and SSS conditions did not differ (\Fsweak[2.7],
\psweak[10]), and the ANOVA showed no effect of structure and no
interaction (all \Fsweak).

}\IGNORE{

NOTHING IN HERE IS UPDATED FOR S#54

pctde \\

overall \\
  struc:  \Fs{52}{.77}{255.30}, \p{384}; \Fi{22}{.57}{45.43}, \p{456} \\
  num:  \Fs[2]{104}{14.98}{250.56}, \p{001}; \Fi[2]{44}{18.84}{58.13}, \p{001} \\
  int:   \Fs[2]{104}{.64}{273.07}, \p{528}; \Fi[2]{44}{.12}{58.13}, \p{881} \\

num pairs \\
  SPSvSSS:  \Fs{52}{17.01}{172.81}, \p{001}; \Fi{23}{25.33}{71.81}, \p{001} \\
  SSPvSSS:  \Fs{52}{.12}{48.30}, \p{728}; \Fi{23}{2.68}{8.01}, \p{115} \\
  SPSvSSP:  \Fs{52}{17.34}{154.73}, \p{001}; \Fi{23}{16.43}{87.94}, \p{001} \\

mismatch \\
  struc:  \Fs{52}{1.52}{626.66}, \p{223}; \Fi{22}{.38}{73.67}, \p{539} \\
  num:  \Fs{52}{17.34}{309.45}, \p{001}; \Fi{22}{15.75}{91.70}, \p{01} \\
  int:  \Fs{52}{.09}{337.24}, \p{757}; \Fi{22}{.05}{91.70}, \p{814} \\

  FSPSvDSPS:  \Fs{52}{.83}{808.41}, \p{367}; \Fi{22}{.03}{149.94}, \p{861} \\
  FSSPvDSSP:  \Fs{52}{2.03}{15.49}, \p{159}; \Fi{22}{1.89}{15.43}, \p{183} \\

  FSPSvFSSP:  \Fs{52}{7.56}{412.11}, \p{01}; \Fi{11}{4.96}{128.57}, \p{05} \\
  DSPSvDSSP:  \Fs{52}{9.72}{234.59}, \p{01}; \Fi{11}{14.80}{54.82}, \p{01} \\

diffs for non-dys or for counts or arcsins: \\
  cntde F1 ssp flat vs desc is marg; all others ns \\
  F2 cntde, pctde, asnde flat n3 mism effect marg; all others ns \\
  cntde F2 flat sps vs ssp marg; all others signif \\
  
}\IGNORE{

However, the more precise test of the hierarchical account is in terms of
mismatch effects, and Figure~\ref{HP7mismatch} shows these effects as a
function of structure and plural position.  The mismatch ANOVA revealed a
main effect of plural position (statistically equivalent to the SPS vs.\
SSP comparison within the overall ANOVA's main effect of number), but as in
the overall ANOVA, the main effect of structure was not significant
(\Fsweak[1.7], \psweak[15]), and, critically, neither was the interaction
(\Fsweak).

}Figure~\ref{HP7mismatch} shows the mismatch effects as a function of
structure and plural position.  N2 plurals produced larger mismatch effects
than N3 plurals (\Fs{53}{16.04}{313.04}\showp{, \p{001}};
\Fi{22}{13.92}{93.22}\showp{, \p{01}}), but the main effect of structure
was not significant (\Fsweak[1.7], \psweak[15]), and, critically, neither
was the interaction (\Fsweak).

%\insertnote{Figure~\ref{HP7mismatch}}

\begin{figure}[tb]
    \makebox[\textwidth][c]{\framefig{\includegraphics[width=.75\textwidth]{HP7mismatch}}}

\caption{Experiment~1 grand mean mismatch error rates by structure and
mismatch position.  Error bars show $\pm1\,$\itt{SEM}, computed from the
analyses by participants.}

\label{HP7mismatch}
\end{figure}

\paragraph{Uninflected and miscellaneous responses.} There were no reliable
main effects or interactions in either the uninflected response count
analyses (all \Fsweak) or the miscellaneous response count analyses (all
\Fsweak[2.2], \psweak[15]).\IGNORE{ this next sentence could go here, but
it's out of place; work it into the discussion (or, maybe better, the GD)
if needed: The lack of any difference in miscellaneous responses for
descending versus flat structures suggests that the two structure types did
not differ in overall complexity or difficulty.}\IGNORE{

NOTHING IN HERE IS UPDATED FOR S#54

old stuff including overall analyses:

\paragraph{Uninflected responses.} Neither the overall nor the
mismatch ANOVAs revealed any main effects or interactions in
uninflected response counts (all \Fsweak).

\paragraph{Miscellaneous responses.} In the overall miscellaneous response
ANOVA, there was a main effect of number (\Fs[2]{104}{9.21}{.56}\showp{,
\p{001}}; \Fi[2]{44}{6.59}{3.48}\showp{, \p{01}}), because the SSS
condition produced fewer miscellaneous responses than either the SPS
(\Fs{52}{8.50}{.27}\showp{, \p{01}}; \Fi{23}{10.59}{1.89}\showp{, \p{01}})
or the SSP condition (\Fs{52}{16.46}{.30}\showp{, \p{001}};
\Fi{23}{16.37}{2.69}\showp{, \p{01}}), while the latter two did not differ
(\Fsweak[2], \psweak[15]).  There was no effect of structure nor an
interaction (all \Fsweak[1.5], \psweak[20]).  Consistent with the SSS
condition creating the main effect of number in the overall analysis, the
mismatch effect ANOVA revealed no main effects and no interaction (all
\Fsweak[2.2], \psweak[15]).  Together, these patterns suggest that having a
plural local noun anywhere in the stimulus made production more difficult,
but this is independent of the critical agreement error patterns.  These 
results also suggest that the descending and flat structures did not 
differ in overall complexity or difficulty.

}\IGNORE{

NOTHING IN HERE IS UPDATED FOR S#54

misc cnts: \\

overall \\
  struc:  \Fs{52}{.08}{.60}, \p{772}; \Fi{22}{.01}{12.07}, \p{893} \\
  num:  \Fs[2]{104}{9.21}{.56}, \p{001}; \Fi[2]{44}{6.59}{3.48}, \p{01} \\
  int:  \Fs[2]{104}{1.42}{.66}, \p{245}; \Fi[2]{44}{1.20}{3.48}, \p{310} \\

num pairs \\
  SPSvSSS:  \Fs{52}{8.50}{.27}, \p{01}; \Fi{23}{10.59}{1.89}, \p{01} \\
  SSPvSSS:  \Fs{52}{16.46}{.30}, \p{001}; \Fi{23}{16.37}{2.69}, \p{01} \\
  SPSvSSP:  \Fs{52}{1.92}{.28}, \p{171}; \Fi{23}{.78}{5.95}, \p{384} \\

mismatch \\
  struc:  \Fs{52}{.36}{1.56}, \p{548};\Fi{22}{.77}{3.25}, \p{388} \\
  num:  \Fs{52}{1.92}{.55}, \p{171}; \Fi{22}{.79}{5.88}, \p{381} \\
  int:  \Fs{52}{2.10}{.81}, \p{153}; \Fi{22}{1.28}{5.88}, \p{270} \\

}

\subsection{Discussion}

These error patterns, and especially the lack of an interaction, provide
strong evidence in favor of a linear distance to the head account of
agreement errors and against a hierarchical distance account: The linear
distance account correctly predicted a larger mismatch effect for N2 than
for N3 mismatches for both descending and flat structures, and it correctly
predicted that the size of this difference would be equal for the two
structures.  The hierarchical account also predicted a larger mismatch
effect for N2 than for N3 mismatches, at least in the descending structure
cases (following Franck et al., 2002), but it also required an
interaction\IGNORE{ in the mismatch analyses}, with a larger difference in
mismatch effects for the descending than the flat cases.

In addition, these results indicate that the semantic integration confound
in Franck et al.'s (2002) stimuli is not solely responsible for the pattern
they found, because semantic integration alone cannot account for the
larger N2 than N3 mismatch effect.  Integration norming showed that N1-N2
integration was matched across structures, as was N1-N3 integration, and
N1-N3 integration was higher than N1-N2 integration in both structures.  An
account of the results based solely on semantic integration thus predicts a
larger N3 than N2 mismatch effect in both structures (as well as no
interaction).  However, Experiment~1's results do not rule out the Solomon
and Pearlmutter (2004b) proposal that semantic integration influences
agreement computation in combination with other factors; because semantic
integration was not varied across structures, this proposal is consistent
with the lack of an interaction in Experiment~1, and the difference between
N1-N2 and N1-N3 integration may have contributed (equally) to the
difference between the N2 and N3 mismatch effects in the two structures.
Experiment~2 was designed to examine this possibility and how semantic
integration might interact with linear distance to the head.

\section[Experiment~2]{\center Experiment~2}

Experiment~1 suggested that linear distance back to the head noun, rather
than hierarchical distance, is a critical factor modulating mismatch
effects, at least in NP PP PP preambles; but it also left open the
possibility that semantic integration has an influence.  Solomon and
Pearlmutter (2004b) showed that the size of the mismatch effect for NP PP
preambles was partially determined by how semantically integrated the head
noun and local noun were within the subject NP, irrespective of a local
noun's distance back to the head noun and of hierarchical distance.
Experiment~2 was thus designed to test whether linear distance to the head
noun and semantic integration combine to influence agreement error
production, using flat structures like~\exref{flat} that controlled
hierarchical distance.  Under a combined linear distance and semantic
integration account, the likelihood of interference would be a function of
whether the interfering element was within the scope of planning of the
head noun; only local nouns planned close enough in time to the head would
be likely to create mismatch effects, with both decreased head-local linear
distance and increased head-local semantic integration increasing the
chance of overlap in planning.  To our knowledge, the effect of scope
properties on agreement computation has not previously been investigated.

In Experiment~2, each preamble had an NP PP PP structure, and the number of
N2 and N3 was varied, as in~\exref{HP2}.  One PP (e.g., \ext{with the torn
page(s)}) was designed to be highly integrated with the head noun, while
the other (e.g., \ext{by the red pen(s)}) was designed to be weakly
integrated, and examining both possible PP orderings allowed linear order
and semantic integration to be manipulated orthogonally.  The stimuli
equated hierarchical distance between the head noun (and thus also the
verb) and each of the two local nouns by ensuring that both PPs modified
the head.

\begin{myexample}
\label{HP2}
\begin{examples}
    
    \item \label{HP2-early-sss} The book with the torn page by the red
    pen \hfill (SSS)\hspace{12em}
    
    \item \label{HP2-early-sps} The book with the torn pages by the 
    red pen \hfill (SPS)\hspace{12em}

    \item \label{HP2-early-ssp} The book with the torn page by the red
    pens \hfill (SSP)\hspace{12em}
    
    \item \label{HP2-early-spp} The book with the torn pages by the red
    pens \hfill (SPP)\hspace{12em}
    
    \item \label{HP2-late-sss} The book by the red pen with the torn
    page \hfill (SSS)\hspace{12em}

    \item \label{HP2-late-sps} The book by the red pens with the torn 
    page \hfill (SPS)\hspace{12em}

    \item \label{HP2-late-ssp} The book by the red pen with the torn 
    pages \hfill (SSP)\hspace{12em}

    \item \label{HP2-late-spp} The book by the red pens with the torn 
    pages \hfill (SPP)\hspace{12em}
    
\end{examples}
\end{myexample}

If only linear distance to the head noun is responsible for agreement error
rates in these stimuli, the N2 mismatch effect should be larger than the N3
mismatch effect in both the early- and late-integrated
versions~(\ref{HP2}a--d and~\ref{HP2}e--h, respectively), as in
Experiment~1, because N2 would always be planned closer to N1 than N3 would
be.

Although Experiment~1's results cannot be explained by semantic
integration, agreement error effects in the Experiment~2 stimuli still
might only show effects of integration, in which case, collapsing over
integration version, the N2 and N3 mismatch effects should be equal,
because the mismatch effect of a given noun should be the same regardless
of where the noun appears linearly within the stimulus.  However, the N2
and N3 mismatch effects within each integration condition should differ:
The N2 mismatch effect should be larger than the N3 mismatch effect for
early-integrated cases~(\ref{HP2}a--d), because the more integrated noun
(N2) should be planned with N1, while N3 should be planned later.  For
late-integrated cases~(\ref{HP2}e--h), the pattern should reverse, because
N3 is the noun more tightly integrated with N1.

The scope of planning alternative, a combined effect of linear distance to
the head and semantic integration, predicts that linear distance to the
head noun will partially determine the timing of planning of nouns within
the phrase, but semantic integration should shift the relative planning
time of more and less integrated nouns as well.  Figure~\ref{timeline}A
shows the timing of planning of nouns according to the order in which they
are to be produced (this corresponds to the predictions of linear distance
to the head alone), and Figure~\ref{timeline}B includes the shifting of the
timing of planning due to semantic integration.  In early-integrated
cases~(\ref{HP2}a--d), because N2 is more integrated with N1, it would be
planned at roughly the same time as N1; and because N3 is less integrated
with N1, N3 would be planned later.  This predicts a large N2 mismatch
effect and a very small N3 mismatch effect.  In late-integrated
cases~(\ref{HP2}e--h), because N2 is not very integrated with N1, it would
be planned later; and because N3 is more integrated with N1, it would be
planned sooner.  Thus N2 and N3 should be planned at roughly the same time
and relatively late after N1.  This predicts that the N2 and N3 mismatch
effects should be about equal and fairly small.

%\insertnote{Figure~\ref{timeline}}

\begin{figure}[tb]

    \makebox[\textwidth][c]{\framefig{\includegraphics[width=.75\textwidth]{timeline}}}

    \caption{Timelines depicting the predicted timing of planning of nouns
    in the Experiment~2 early-integrated and late-integrated stimuli.
    Panel~A shows the nouns planned according to the order in which they
    are to be produced, corresponding to the predictions of the linear
    distance to the head account.  Panel~B shows the timing of planning of
    the nouns after semantic integration shifts their relative timing,
    corresponding to the scope of planning account.}

\label{timeline}
\end{figure}

\subsection{Method}

\paragraph{Participants.} One hundred five Northeastern University
students and community members participated in the on-line experiment, but
the data from one participant were lost because of a CD recording failure.

\paragraph{Materials and design.} Forty stimulus items
like~\exref{HP2}, chosen from a candidate set of~60, were used as
critical items.  All preambles consisted of a head NP (always
\ext{The} and a singular head noun, N1; e.g., \ext{book}
in~\exref{HP2}) followed by two PPs, each of which consisted of a
preposition, the determiner \ext{the}, an adjective or modifier noun,
and a local noun (N2 or N3), as in the Experiment~1 stimuli.  One PP
always described an attribute of the head noun using the preposition
\ext{with} (e.g., \ext{with the torn page}), while the other PP
specified a location for the head noun and used a locative preposition
(e.g., \ext{by the red pen}).  The eight different versions of an item
were created by varying N2 and N3 number and PP order, as shown
in~\exref{HP2}.  The versions with the attribute PP first were the
early-integrated versions~(\ref{HP2}a--d), and those with the
attribute PP last were the late-integrated versions~(\ref{HP2}e--h).
All nouns in the stimuli were inanimate and had regular plural forms,
and each noun's conceptual number matched its grammatical number.  The
full set of stimuli is shown in Appendix~B.

The~40 critical stimuli were combined with~80 of the fillers from
Experiment~1 (all but the~8 SPP fillers used in Experiment~1 to balance the
critical items' SSS, SPS, and SSP cases) to form~8 counterbalanced
presentation lists, each containing all the fillers and exactly one version
of each of the critical items.  One list was seen by~14 participants while
the other seven were seen by~13 participants each.

\paragraph{Stimulus norming.} Semantic integration and attachment were
normed similarly to Experiment~1.  In addition to the~60 candidate
stimuli,~33 NP PP PP fillers were included.  The fillers had a
descending hierarchical structure and did not have adjectives within
the PPs; but they varied local noun number as in the candidate
stimuli, and they included a range of levels of semantic integration
between N1, N2, and N3.

Semantic integration ratings were obtained from~186 participants, using the
same procedures and instructions as in Experiment~1.  The~24 different
versions of the~60 candidate stimuli (2 PP orders $\times$ 4 number
conditions $\times$ 3 possible rating pairs) were combined with the~33
fillers in~24 counterbalanced lists, such that each list included exactly
one version of each item.  The~93 preambles in each list were presented
over~6 printed pages, which were randomized separately for each
participant.  Each participant rated the stimuli in one list, yielding~7--8
ratings for nearly all versions of all critical stimuli~(3 different items
had only~6 ratings for one of their versions).

Table~\ref{HP2-dat} shows mean ratings by condition and rating pair for
the~40 critical stimuli used in the on-line experiment.  These were
analyzed in a 2 (PP order) $\times$ 2 (N2 number) $\times$ 2 (N3 number)
$\times$ 3 (rating pair) ANOVA, which showed the desired interaction
between PP order and rating pair (\Fval{2}{78}{1025}, \MSe{.97}\showp{,
\p{001}}): N1-N2 integration was greater than N1-N3 integration in the
early-integrated stimuli (\Fval{1}{39}{1137}, \MSe{.22}\showp{, \p{001}}),
while this pattern reversed in the late-integrated stimuli
(\Fval{1}{39}{1140}, \MSe{.21}\showp{, \p{001}}), and N2-N3 integration was
equal for early- and late-integrated cases (\Fweak).  Paired tests also
confirmed that, as intended, N1-N2 integration was greater for early- than
for late-integrated versions (\Fval{1}{39}{1157}, \MSe{.22}\showp{,
\p{001}}), N1-N3 integration was greater for late- than for
early-integrated versions (\Fval{1}{39}{1016}, \MSe{.24}\showp{, \p{001}}),
and integration did not differ for the same local noun in different
positions (i.e., N1-N2 for early-integrated vs.\ N1-N3 for late-integrated,
and N1-N2 for late-integrated vs.\ N1-N3 for early-integrated; both
\Fsweak).

%\insertnote{Table~\ref{HP2-dat}}

{
%\small  %%if we need to set the table in 11pt instead of 12pt
\settowidth{\maxcolwidth}{Uninflected}
%\settowidth{\maxcolwidth}{999 (99)} %% use this version with Uninfl 
                                     %% col head
\settowidth{\Attwidth}{Attachment}

\setstretch{1}  %% single-spacing for the table

\ctable[
        caption = {\raggedright Experiment~2 Mean Semantic Integration Ratings, Mean
            Attachment Preferences, and Response Counts by Condition},
        cap = {},
        captionskip=2ex,
        label = HP2-dat,
        sideways, 
        star
       ]{l l T{4.6} T{4.6} T{4.6} T{4.6} S@{}S@{}S@{}T{5.0}}{
        \tnote[]{\normalsize\spacer{0}{4}\textit{Note}.  Conditions are
            indicated by noun number (N1, N2, N3), with S = singular and P =
            plural.  Semantic integration rating scale was 1 (loosely linked) to 7
            (tightly linked).  For semantic integration ratings and attachment,
            means are computed by-items and standard deviations are in parentheses.
            For response counts, dysfluency counts are in parentheses.  Misc =
            Miscellaneous.}}{
        \hline
        \spacer{-2}{6} & & \multicolumn{3}{c}{Semantic Integration Rating} & 
                     & \multicolumn{4}{c}{Response Count} \\ \cline{3-5} \cline{7-10}
                     \multicolumn{2}{l}{\spacer{-2}{6}Condition} & 
                     \multicolumn{1}{c}{N1--N2} &  \multicolumn{1}{c}{N1--N3} & 
                      \multicolumn{1}{c}{N2--N3} & 
                      \multicolumn{1}{c}{\raisebox{0ex}[0ex][0ex]{\parbox[b]{\Attwidth}{\center \%N1 Attachment}}} & 
                      \multicolumn{1}{c@{}}{\makebox[\maxcolwidth][c]{Error}} &
                      \multicolumn{1}{c@{}}{\makebox[\maxcolwidth][c]{Correct}} &
                      \multicolumn{1}{c@{}}{\makebox[\maxcolwidth][c]{Uninflected}} &
                      \multicolumn{1}{c}{\makebox[\maxcolwidth][c]{Misc}} \\ \hline
                      \multicolumn{4}{l}{\spacer{-2}{6}Early-integrated} \\
                      SSS & & 5.74!(.74) & 2.24!(.62) & 2.08!(.59) & 96.4!(8.21) & 2.(1) & 322.(45) & 109.(15) & 91 \\
                      SPS & & 5.70!(.56) & 2.16!(.62) & 2.00!(.45) & 98.9!(6.05) & 27.(8) & 270.(33) & 95.(11) & 132 \\
                      SSP & & 5.63!(.63) & 2.21!(.60) & 2.04!(.48) & 97.6!(5.49) & 2.(0) & 312.(39) & 91.(10) & 119 \\
                      SPP & & 5.75!(.60) & 2.12!(.54) & 2.05!(.57) & 98.0!(5.06) & 36.(5) & 253.(29) & 108.(16) & 128 \\
                      \spacer{-2}{6}\textit{M} & & 5.70!(.63) & 2.18!(.59) & 2.04!(.52) & 97.7!(5.63) \\ \hline
                      \multicolumn{4}{l}{\spacer{-2}{6}Late-integrated} \\
                      SSS & & 2.25!(.65) & 5.63!(.66) & 1.98!(.54) & 97.7!(3.79) & 1.(1) & 318.(54) & 105.(20) & 100 \\
                      SPS & & 2.03!(.47) & 5.71!(.64) & 2.00!(.62) & 98.9!(2.74) & 15.(5) & 276.(42) & 96.(15) & 138 \\
                      SSP & & 2.18!(.58) & 5.57!(.75) & 2.18!(.62) & 97.7!(3.98) & 8.(0) & 313.(35) & 102.(17) & 102 \\
                      SPP & & 2.19!(.56) & 5.81!(.80) & 2.17!(.53) & 99.2!(3.29) & 32.(3) & 290.(42) & 90.(16) & 113 \\
                      \spacer{-2}{6}\textit{M} & & 2.16!(.57) & 5.68!(.71) & 2.08!(.58) & 98.4!(3.52) \\ \hline
                      \multicolumn{2}{l}{\spacer{-2}{6}Grand Mean} & 3.93!(1.87) &
                      3.93!(1.87) & 2.06!(.55) & 98.0!(5.12) \\ \hline
         }
}

The overall ANOVA on integration ratings also revealed a main effect of
rating pair (\Fval{2}{78}{908}, \MSe{.41}\showp{, \p{001}}), which arose
because the N2-N3 pair was overall less semantically integrated than the
N1-N2 or N1-N3 pair; given the modification structure of the stimuli and
the nature of the PP order by rating pair interaction described above, this
was expected.  However, in addition to these very strong effects, the
overall ANOVA also yielded two weaker results involving local noun number:
First was a reliable interaction between PP order, N2 number, and rating
pair (\Fval{2}{78}{3.51}, \MSe{.21}, \p{05}), which arose because the PP
order by rating pair interaction described above was stronger when N2 was
plural than when it was singular: The N1-N2 versus N1-N3 difference for
early-integrated was 3.58 for plural N2 and 3.48 for singular N2; the
corresponding differences for late-integrated were -3.65 and -3.38.  The
second result was a marginally significant N2 number by N3 number
interaction (\Fval{1}{39}{3.13}, \MSe{.20}), which arose because the
difference in integration ratings between the SPS and SPP conditions was
slightly larger than the difference between the SSP and SSS conditions.  We
will return to both of these unexpected results below, with the discussion
of the results of the on-line experiment.  No other main effects nor
interactions appeared in the overall ANOVA (all \Fsweak[2.6],
\psweak[10]).\IGNORE{

n2 x n3: \Fval{1}{39}{3.13}, \MSe{.20}, \p{10} \\
pair: \Fval{2}{78}{908.63}, \MSe{.41}, \p{001} \\
ord x pair: \Fval{2}{78}{1025.65}, \MSe{.97}, \p{001} \\
ord x n2 x pair: \Fval{2}{78}{3.51}, \MSe{.21}, \p{05} \\

all other effects: \Fsweak[2.6], \psweak[10] \\

}\IGNORE{

\NOTE{from Maureen's Master's; adjust for ms}

To analyze semantic integration rating data, a 2 (integration condition) X
2 (N2 number) X 2 (N3 number) X 3 (pair) ANOVA was performed.  The ANOVA
showed that there was a main effect of pair such that the N2-N3 pair was
less integrated than the N1-N2 and N1-N3 pairs (F(2,78) = 908.63, MSe =
.41, p < .001), which does not affect any predictions of the proposed
experiment.  There was an interaction of integration condition and pair
(F(2,78) = 1025.65, MSe = .97, p < .001) such that the N2-integrated N1-N2
pair and the N3-integrated N1-N3 pair were rated as more integrated than
the N2-integrated N1-N3 pair and the N3-integrated N1-N2 pair, which was
critical for the predictions of the experiment, because semantic
integration was manipulated by the ordering of the with- and locative-PPs.
There was also an interaction of integration condition, N2 number, and pair
(F(2,78) = 3.52, MSe = .21, p < .05), such that when N2 was plural in the
N2-integrated version, the N1-N3 pair was rated as less integrated than
when N2 was singular, and when N2 was plural in the N3-integrated version,
the N1-N3 pair was rated as more integrated than when N2 was singular.
This three-way interaction was unexpected and will be addressed below.  The
interaction of N2 and N3 number was marginally significant, such that the
difference in integration ratings between the SPS and SPP conditions was
slightly larger than the difference between the SSP and SSS conditions
(F(1,39) = 3.14, MSe = .21, p = .08).  No other main effects or
interactions were significant (Fs < 3, ps < .15).

Planned paired comparisons were performed on the semantic integration means
presented in Table 1.  To control for effects of semantic integration
across levels of linear distance to the head, items were chosen that best
equated semantic integration ratings of the with-PPs in the N2- and
N3-integrated conditions and semantic integration ratings of the
locative-PPs in the N2- and N3-integrated conditions.  A paired comparison
showed that there was no difference between the ratings of the
N2-integrated N1-N2 pair and the N3-integrated N1-N3 pair (F < 1).
Similarly, there was no difference between the ratings of the N2-integrated
N1-N3 pair and N3-integrated N1-N2 pair (F < 1).  It was also important
that the with-PPs and the locative-PPs differed in semantic integration to
obtain the desired differences in integration across the N2- and
N3-integrated conditions.  Paired comparisons confirmed that the
integration ratings of the N1-N2 pair differed in the N2-integrated and
N3-integrated conditions, (F(1,39) = 1157.32, MSe = .21, p < .001), and the
integration ratings of the N1-N3 pair differed in the N2-integrated and
N3-integrated conditions, (F(1,39) = 1016.11, MSe = .24, p < .001).

In sum,   Semantic integration ratings show
that the with-PP is more integrated with the head NP than the locative-PP
in both the N2- and N3-integrated conditions, and that the with-PP and
locative-PP integration ratings are equal regardless of order, as was
desired.  This resulted in matched overall N1-N2 and N1-N3 integration
ratings when collapsing over the N2- and N3-integrated conditions.  This
selection process served to equate hierarchical distance to the head noun
between N2 and N3, while manipulating semantic integration.

\NOTE{end of stuff from Master's}

\TODO{adjust/integrate this} In the early-integrated versions, the N1-N2
pair was more integrated than the N1-N3 pair, while in the late-integrated
versions, the N1-N3 pair was more integrated than the N1-N2 pair (Fs >
1000, ps < .001).  Also, the more integrated nouns (page(s) in~\exref{HP2})
were equally integrated regardless of integration version, as were the less
integrated nouns (pen(s) in~\exref{HP2}) (Fs < 1).

}

To ensure that all critical stimuli had a flat structure, attachment
of the second PP was normed by~150 participants\IGNORE{ran 152,
dropped one non-native English speaker and one person who answered
with N3 repeatedly}.  The procedure was slightly different from
Experiment~1's attachment norming, in that the final PP of each
preamble was underlined, and participants were instructed to write the
word that the underlined portion of the preamble ``gives more
information about''.  The~8 versions of each candidate stimulus were
counterbalanced across~8 lists in combination with the~33 fillers, and
the preambles of each list were presented over~6 printed pages, which
were randomized separately for each participant.  This yielded~18--19
responses for each version of each critical item.

As in Experiment~1, the responses were coded for whether they referred to
N1 or to N2 (excluding unclear or uninterpretable responses, $<1\%$ of the
total), and Table~\ref{HP2-dat} shows mean preference for attachment to N1
by condition for the~40 critical stimulus items.  Attachment of the second
PP was strongly to N1 in all conditions, as desired, and a 2 (PP order)
$\times$ 2 (N2 number) $\times$ 2 (N3 number) ANOVA on the \%N1 attachment
data revealed only that plural N2 cases (98.8\%) were slightly more
strongly N1-attached than singular N2 cases (97.3\%; \Fval{1}{39}{4.91},
\MSe{32.24}\showp{, \p{05}}).  No other main effects and no interactions
were significant (all \Fsweak[1.8], \psweak[15]).

\paragraph{Apparatus, procedure, and scoring.} The apparatus, procedure, and
response scoring for Experiment~2 were identical to Experiment~1, except
that there were~120 trials.

\subsection{Results}

Across all critical trials, there were~2,354 correctly-inflected
responses,~123 agreement errors,~796 uninflected responses,~923
miscellaneous cases, and~1 trial with no response.  The remaining~3 trials
were lost due to a recording failure for one subject.  Table~\ref{HP2-dat}
shows the counts of each analyzed response type by condition.  All analyses
were conducted as in Experiment~1, except that the (within-items)
integration factor replaced Experiment~1's (between-items) structure factor
in the ANOVAs.\IGNORE{

NOTHING IN HERE IS UPDATED FOR S#203+, except diffs for mismatch stuff

pctde \\

overall \\
  integ:  \Fs{98}{2.84}{247.68}, \p{10}; \Fi{39}{1.60}{76.99}, \p{213} \\
  2num:  \Fs{98}{49.07}{228.98}, \p{001}; \Fi{39}{33.51}{142.73}, \p{001} \\
  int x n2:  \Fs{98}{11.36}{183.87}, \p{01}; \Fi{39}{4.40}{81.95}, \p{05} \\
  3num:  \Fs{98}{7.70}{209.21}, \p{01}; \Fi{39}{8.31}{114.22}, \p{01} \\
  int x n3:  \Fs{98}{3.72}{192.15}, \p{10}; \Fi{39}{.81}{70.96}, \p{372} \\
  n2 x n3:  \Fs{98}{1.45}{226.06}, \p{231}; \Fi{39}{4.53}{84.87}, \p{05} \\
  int x n2 x n3:   \Fs{98}{.00}{203.20}, \p{974}; \Fi{39}{.33}{55.68}, \p{564} \\

mismatch \\
  integ:  \Fs{98}{.86}{230.01}, \p{356}; \Fi{39}{.05}{63.91}, \p{812} \\
  num:  \Fs{98}{10.32}{209.93}, \p{01}; \Fi{39}{13.17}{55.79}, \p{01} \\
  int:  \Fs{98}{13.92}{188.67}, \p{001}; \Fi{39}{4.42}{80.00}, \p{05} \\

  ISPSvUSPS:  \Fs{98}{8.11}{263.01}, \p{01}; \Fi{39}{1.86}{114.95}, \p{180} \\
  ISSPvUSSP:  \Fs{98}{4.44}{155.67}, \p{05}; \Fi{39}{4.93}{28.96}, \p{05} \\

  ISPSvISSP:  \Fs{98}{20.28}{235.87}, \p{001}; \Fi{39}{12.68}{83.14}, \p{01} \\
  USPSvUSSP:  \Fs{98}{.06}{162.73}, \p{795}; \Fi{39}{.65}{52.64}, \p{424} \\

  ISPSvUSSP:  \Fs{98}{5.04}{364.60}, \p{05}; \Fi{39}{5.33}{78.98}, \p{05} \\
  ISSPvUSPS:  \Fs{98}{7.00}{75.34}, \p{01}; \Fi{39}{7.79}{40.72}, \p{01} \\

diffs for non-dys or for counts or arcsins: \\
  
overall int F1 pctde marg, all others ns \\
overall int x n2 cnte F2 ns, cnte F1 & cntde F2 & pcte F2 marg, all others signif \\
overall int x n3 mixed (all over the place) \\
overall n2 x n3  mixed \\

overall ssp vs sss F1 cntde, pctde, asnde marg, all others signif \\
overall sps vs ssp F2 asne marg, all others signif \\

UPDATED FOR S#203+:

mismatch 2x2 (no diffs) \\

mismatch pairs \\
  ISPS vs ISSP, USPS vs USSP, ISSP vs USPS --- all agree \\
  ISPS vs USPS F1 pctde & asnde signif; cntde & cnte marg, pcte & asne ns; all F2 ns \\
  ISSP vs USSP all signif except F1cntde marg \\
  ISPS vs USSP (integ word): dys cases signif; others ns except F1cnte signif \\

}

\paragraph{Agreement errors.} Figure~\ref{HP2mismatch} shows N2 and N3
mismatch effects for each integration condition.  The mismatch effect ANOVA
showed no main effect of integration (\Fsweak), but N2 plurals created
larger mismatch effects than N3 plurals (\Fs{104}{13.74}{237.78}\showp{,
\p{001}}; \Fi{39}{17.57}{65.84}\showp{, \p{001}}).  Furthermore, plural
position and integration interacted (\Fs{104}{9.31}{203.33}\showp{,
\p{01}}; \Fi{39}{4.13}{78.65}\showp{, \p{05}}), because the N2 mismatch
effect was larger than the N3 mismatch effect for early-integrated stimuli
(\Fs{104}{22.24}{227.88}\showp{, \p{001}}; \Fi{39}{15.76}{85.94}\showp{,
\p{001}}), while the two mismatch effects did not differ for
late-integrated stimuli (\Fsubweak{1}; \Fi{39}{2.18}{58.55}, \p[>]{10}).

%\insertnote{Figure~\ref{HP2mismatch}}

\begin{figure}[tb]
\makebox[\textwidth][c]{\framefig{\includegraphics[width=.75\textwidth]{HP2mismatch}}}

\caption{Experiment~2 grand mean mismatch error rates by integration and
mismatch position.  Error bars show $\pm1\,$\itt{SEM}, computed from the
analyses by participants.}

\label{HP2mismatch}
\end{figure}

The other four pairings of the four mismatch conditions were also tested
with planned comparisons: The early-integrated N2 plural condition
generated larger mismatch effects than the late-integrated, though this was
reliable only in the analysis by participants
(\Fs{104}{4.42}{292.87}\showp{, \p{05}}; \Fi{39}{1.76}{117.97}, \p[>]{15};
the analysis of error counts by participants was also only marginal, and
the analysis by participants excluding dysfluencies did not reach
significance).  The pattern reversed for N3 plurals, where the
late-integrated condition showed larger mismatch effects than the
early-integrated (\Fs{104}{4.43}{147.07}\showp{, \p{05}};
\Fi{39}{4.68}{26.25}\showp{, \p{05}}; the difference was only marginal in
the analysis of error counts by participants).  The tests comparing the
same phrases in different linear positions also showed differences: The
early-integrated N2 plural condition created larger mismatch effects than
the late-integrated N3 plural condition (\Fs{104}{5.96}{349.55}\showp{,
\p{05}}; \Fi{39}{8.34}{79.32}\showp{, \p{01}}; this was non-significant
when dysfluencies were excluded), and the late-integrated N2 plural
condition created larger mismatch effects than the early-integrated N3
plural condition (\Fs{104}{9.92}{124.84}\showp{, \p{01}};
\Fi{39}{9.62}{52.09}\showp{, \p{01}}).

\paragraph{Uninflected responses.} There were no reliable main effects or
interactions for uninflected responses (all \Fsweak).

\paragraph{Miscellaneous responses.} N2 mismatches generated more
miscellaneous responses than N3 mismatches (\Fs{104}{6.67}{.86}\showp{,
\p{05}}; \Fi{39}{7.88}{1.90}\showp{, \p{01}}), but there was no main effect
of integration (\Fsweak) and no interaction (\Fsweak[1.4], \psweak[20]).

\subsection{Discussion}

Experiment~2 examined how effects of linear distance to the head and of
semantic integration might be involved in determining the size of mismatch
effects.  Because N2 was always linearly closer to N1 than N3 was, linear
distance alone predicted that the N2 mismatch effect should be uniformly
larger than the N3 mismatch effect.  While this difference appeared for the
early-integrated stimuli, the N2 and N3 mismatch effects did not differ for
the late-integrated stimuli.  The semantic integration account predicted
that the mismatch effects for the more integrated cases (N2 in the
early-integrated stimuli, N3 in late-integrated) should not differ and
should both be larger than for the less integrated cases (N3 in
early-integrated, N2 in late-integrated; these also should not differ from
each other).  Instead, the early-integrated N2 mismatch effect was larger
than the late-integrated N3 mismatch effect (at least when dysfluencies
were included), and the late-integrated N2 mismatch effect was larger than
the early-integrated N3 mismatch effect, showing that the more integrated
cases did differ, as did the less integrated cases; and the two
late-integrated cases did not differ in mismatch effects despite differing
in semantic integration.  Semantic integration alone also predicted no main
effect of plural position (the N2 and N3 mismatch effects should have been
equal).  Thus, neither linear distance to the head noun nor semantic
integration on its own is sufficient to explain the pattern of effects.

Instead, error rates showed a pattern reflecting a combination of linear
distance to the head and semantic integration, which fits with the scope of
planning account.  The scope account in general suggests that mismatch
effects will be more likely to occur for local nouns planned in closer
proximity to the head noun, and factors that affect planning proximity, in
the combination case including both linear distance from the head and
semantic integration, can thus affect mismatch error rates.  More
specifically, this account predicts a relatively large difference in
mismatch error rates between the two early-integrated cases, because N2 is
both more integrated and linearly closer to the head than N3; and it
predicts a relatively small difference between the two late-integrated
cases, because N2 is linearly closer to the head than N3 is, but N3 is more
integrated with the head than N2 is.  Thus semantic integration reinforces
the linear distance difference in the early-integrated cases, but it
counteracts the linear distance difference in the late-integrated cases,
and this is the pattern seen in mismatch effects: For early-integrated
stimuli, the N2 mismatch effect was larger than the N3 mismatch effect, but
for late-integrated stimuli, they did not differ.

Although the results point to a combination of linear distance to the head
and semantic integration as responsible for the mismatch effect pattern,
and neither of these alone is sufficient to explain the full pattern of
results, the planned comparisons also provided evidence for the influence
of each factor when the other was controlled: In particular, the
comparisons of the early- versus late-integrated cases within each plural
position (N2, N3) are a direct test of the effect of semantic integration
when linear distance was controlled.  These comparisons revealed that the
more integrated noun at each plural position generated larger mismatch
effects than the corresponding less integrated noun (early-integrated N2
vs.\ late-integrated N2; late-integrated N3 vs.\ early-integrated N3).
Direct tests of the effect of linear distance when semantic integration was
controlled come from the comparisons in which the same PP occurred in the
two different linear positions: Within the two attribute PPs
(early-integrated N2 vs.\ late-integrated N3), the N2 cases generated a
larger mismatch effect; and the same was true for the two locative cases
(late-integrated N2 vs.\ early-integrated N3).  Because the scope of
planning account incorporates effects of both linear distance to the head
and semantic integration, it predicts these results as well.

\IGNORE{
\NOTE{this para was in original HP2 ms but apparently not in this 
one; we may want it or some version of it, either here or in GD:} --- keep 
it here}

The evidence that both linear distance to the head and semantic integration
are relevant for a general scope of planning account also raises the issue
of exactly how they combine.  While they can both be seen as influencing
the timing of planning, whether they are independent influences or not is
an open question.  For example, if the difference in timing of planning for
N2 versus N3 created by linear distance is large enough that N3 is
essentially always planned too long after N1 to interfere with agreement,
then semantic integration might only have an effect for mismatches at N2,
or it might have a much weaker effect at N3 compared to N2.  The relevant
statistical test of the general interaction --- whether the difference
between more- and less-integrated cases was different at the N2 position
versus at the N3 position --- is equivalent to the main effect of
integration, and this effect did not approach significance.  However, the
numerical pattern of mismatch effects did include a larger effect of
integration at N2 compared to N3; future work designed specifically to
investigate this issue should therefore be enlightening.

Before concluding that these patterns are entirely the result of scope of
planning effects, however, we need to consider whether the two unexpected
interactions involving noun number found in the integration rating analyses
--- N2 number $\times$ N3 number and PP order $\times$ N2 number $\times$
rating pair --- might be relevant.  Both of these interactions involved
numerically very small differences relative to the intended integration
manipulations, and the former was in fact only marginal rather than
significant.  The N2 number $\times$ N3 number interaction also does not
account for any of the critical patterns in the mismatch effects, as it
does not involve PP order, and it is linked to the SPP condition ratings,
which were irrelevant for mismatch effects.  But the other of the two
interactions, involving PP order, could in principle be involved in the
pattern of paired comparisons for the same noun in different linear
positions (early-integrated N2 vs.\ late-integrated N3; early-integrated N3
vs.\ late-integrated N2).  These comparisons provided evidence for linear
distance effects when semantic integration was controlled, but the
interaction in the norming shows that integration was not perfectly
controlled, and this might provide an alternative explanation for the
differences in mismatch effects.  In fact, the N1-N2 rating for the
early-integrated SPS condition was slightly higher than the N1-N3 rating
for the late-integrated SSP condition, which correctly predicts the
direction of the corresponding mismatch effect comparison (a larger
mismatch effect for the early-integrated N2 case than for the
late-integrated N3 case).  However, for the other pair, the mismatch effect
continued to follow the linear distance prediction, with the
late-integrated N2 mismatch effect larger than the early-integrated N3
mismatch effect, whereas the interaction in the ratings resulted from the
opposite pattern: The N1-N2 rating for the late-integrated SPS condition
was slightly lower than the N1-N3 rating for the early-integrated SSP
condition.  As a result, the rating interaction pattern diverges from the
mismatch effect pattern.  Thus neither of the unexpected interactions from
the rating analyses seems likely to be responsible for the mismatch
effects.

There was also a main effect of plural position on miscellaneous responses,
with a greater increase in such responses over the SSS baseline for N2
plurals compared to N3 plurals.  This suggests that N2 plural cases were
generally more complex or more difficult than N3 plural cases, at least in
Experiment~2, and it obviously matches the main effect of plural position
in the agreement error analyses, but not the critical interaction.
Nevertheless, if N2 plural cases were generally more difficult to produce
than N3 plural cases, this would provide a potential alternative
explanation for the effects attributed to linear distance to the head.
Experiment~2's results do not rule out this possibility, but Experiment~1
also showed clear support for effects of linear distance to the head,
without any comparable effects or interactions in miscellaneous errors.
Thus the linear distance to the head account appears to be the more robust
explanation.

\section[General Discussion]{\center General Discussion}

\IGNORE{---no support for hierarchical feature-passing, support for a
limited scope of planning coupled with mem encoding; in combination with
sem integ affecting planning by shifting relative timing}

Together, the results of Experiments~1 and~2 provide no support for
hierarchical feature-passing.  In Experiment~1, when factors known to
influence agreement computation were controlled and hierarchical distance
was manipulated, there was no effect of structure on the size of mismatch
effects; instead, only a local noun's linear proximity to the head noun
affected error rates.  In addition, when structure was controlled in
Experiment~2, a combination of linear distance and semantic integration
controlled error rates.  The results of these experiments point to an
account of agreement computation that relies on memory encoding and a
limited scope of planning.  Mismatch effects are the result of the extent
to which the head noun and interfering local noun(s) are simultaneously
active in memory when the number of the subject NP is being computed.  The
timing of planning of elements within a phrase is determined by the order
in which elements are to be produced, and semantic integration shifts the
relative timing of planning.  Agreement errors are likely to occur when a
number-mismatching local noun is planned within the scope of (i.e., close
in time to) the head noun: Because of the overlap in planning, the nouns
and their corresponding number elements are likely to be active
simultaneously and are likely to interfere at the time when the number
marking of the subject NP is set.

\IGNORE{\subsection{Potential Concerns over Stimulus Structure}}

We discuss the implications of these results further below, but there are
three potential concerns with the current evidence to be considered,
related to the syntactic structure of the stimuli.  The first concern is
that while the offline norming established that the stimuli were
interpreted as having the desired syntactic structures (flat or
descending), initial interpretations during the online experiments could
have been different, particularly in the flat cases.  English comprehenders
prefer to attach new material to more recent over less recent
(otherwise-equivalent) sites (e.g., Frazier \& Clifton, 1996; Gibson,
Pearlmutter, Canseco-Gonzalez, \& Hickok, 1996), and in the Experiment~1
and~2 stimuli this would have resulted in the second PP modifying the more
recent N2 (within the first PP) instead of the less recent N1.  In
Experiment~1, this was the correct interpretation for the descending
stimuli but would have been an incorrect interpretation for the flat
stimuli; it would have been the incorrect interpretation for all the
critical stimuli in Experiment~2, which all had a flat structure.  Although
there was no specific test of attachment of the second PP during the online
experiments, the miscellaneous errors provide a measure of general
difficulty of the preambles; and if flat structures were initially
interpreted incorrectly and then reanalyzed, or if they were simply left
incorrectly attached without reanalysis, yielding semantic anomalies, then
flat structures should have generated more miscellaneous errors than
descending structures in Experiment~1.  But this was not the case: There
was no effect of structure nor an interaction in the analysis of mismatch
effects, and as Table~\ref{HP7-dat} shows, the overall counts of
miscellaneous errors were nearly identical (135 for descending, 130 for
flat; \Fsweak\IGNORE{overall struc Fs<1; int Fs<1.4, ps>.25}).  Coupled
with the relative weakness of the recency effect in English comprehension
for attachments to noun sites in particular (e.g., Cuetos \& Mitchell,
1988; Gibson et al., 1996; Pearlmutter \& Gibson, 2001), the lack of any
difference in mismatch errors for flat versus descending structures
suggests that the second PP was attached as intended (and as normed) for
both structures and in both experiments.\IGNORE{ \NOTE{should we thank Iris
for pointing out this concern in a footnote or something?  I heard it from
other people at CUNY as well, but she was the first to point it out (to me
at least)} -- think we probably don't need the ack in this case -- the
concern about interpreting these structures (PP attachment ambiguity) is a
pretty common issue -- that's why we needed to do the attachment norming}

\IGNORE{---issues w/flat strucs in HP2 (interacting w/semint manip to yield
diff structures in integ vs unint)\ldots anything to discuss
here? \NOTE{wait for complaints}
}

The second possible concern related to the structure of the stimuli is that
the argument/adjunct status of the PPs containing the local nouns might
have varied across conditions, creating confounds.  Solomon and Pearlmutter
(2004b) showed that argument/adjunct status and related structural
properties in their NP PP stimuli could not account for their semantic
integration results, but it still might be involved in the effects here.
Determining the argument/adjunct status of PP modifiers of NPs is often
difficult, but Sch\"{u}tze and Gibson (1999) offer six diagnostics of PP
argumenthood, although they note that individual speakers may disagree in
applying a particular diagnostic to a particular case, and there may be
disagreement across diagnostics even when speakers agree.  When we applied
the diagnostics to the Experiment~2 stimuli, there was in fact strong
agreement (across the authors and the diagnostics) that both the first and
second PPs were adjuncts, eliminating any basis for a confound.  In the
descending stimuli of Experiment~1, both PPs were similarly clearly
adjuncts; and this was also the case for the second PP of the flat stimuli
in Experiment~1.  The only case where there appeared to be any support for
an argument classification was for two of the first PPs in the flat stimuli
(these both involved \ext{of}, as in \ext{The postcard of the roller
coaster with the foreign stamp}), which two of the six diagnostics
(Sch\"{u}tze \& Gibson's ``ordering'' and ``iterativity'') seemed to
classify as arguments.  If these PPs were indeed arguments rather than
adjuncts, then according to syntactic theories which encode the
argument/adjunct distinction in structure (e.g., Carnie, 2005; Chomsky,
1995; Pollard \& Sag, 1994), they would be attached farther from the head
NP node than corresponding adjuncts.  For versions of the hierarchical
feature-passing hypothesis that compute distance based on all intervening
nodes (versus only major categories; see Solomon \& Pearlmutter, 2004b),
this would in turn predict that N2 would be less likely to create errors in
these cases.  But the result of this potential confound is that the
difference between the sizes of the N2 and N3 mismatch effects should be
even smaller for flat cases than for descending cases, because in the flat
cases, N2 will be more distant from the head NP than N3 (the difference in
the size of the mismatch effects will be negative), while in the descending
cases, N2 will be closer to the head NP than N3.  Thus, if it has any
effect, it should be to strengthen the interaction prediction for
hierarchical feature-passing.  There was no hint of such an interaction in
Experiment~1, and thus an argument/adjunct confound cannot explain the
results.\IGNORE{

The second concern related to the stimuli is that all PP modifiers used
were intended be adjuncts of the nouns that they modified, but PP modifiers
of nouns are non-obligatory, which makes determining whether they are
arguments or adjuncts difficult.  Sch\"{u}tze and Gibson (1999) offer six
diagnostics for PP argumenthood, but caution that the diagnostics do not
all provide the same distinction between arguments and adjuncts.  Assuming
that these diagnostics can be used to infer the argument status of a PP
modifier, we tested the PPs in our stimuli to ensure that they were
adjuncts.  The PPs in Experiment~2 were clearly adjuncts by every
diagnostic.  To the extent that there was any disagreement among
diagnostics, some of the first PPs in the flat stimuli of Experiment~1 are
less likely to be adjuncts of the head noun than others.  If these PPs were
indeed arguments rather than adjuncts, under most syntactic theories, this
would increase the hierarchical distance of N2 to the head NP node (e.g.,
Carnie, 2005; Chomsky, 1995; Pollard \& Sag, 1994).  According to the
hierarchical feature-passing account, with all nodes counting toward
hierarchical distance, this would on average make interference from N2 less
likely than interference from N3.  Thus, the interaction of structure and
plural position predicted by the hierarchical feature-passing account in
Experiment~1 would only be larger given that the interference patterns of
N2 and N3 would be predicted to be in opposite directions in the two
structural conditions.  Given that an interaction was not found in
Experiment~1, this concern is unlikely to be a problem for the scope of
planning account (for a discussion of argument/adjunct status in relation
to semantic integration effects, see Solomon \& Pearlmutter, 2004b).
}

The final concern related to the structure of the stimuli is that in some
syntactic theories (e.g., Chomsky, 1995\IGNORE{; \TODO{cite 1--2 other
theories/papers}}), flat structures are ruled out; typically this is
because such theories enforce binary branching (Kayne, 1984), stipulating
that no syntactic node may have more than two daughters.  The evidence for
a binary-branching requirement is largely theory-internal\IGNORE{ (though
cf.\ \TODO{Iris' suggested paper} and responses)}, depending primarily on
what other assumptions a theory holds about the relationship between
syntactic structure, meaning, and discourse modification structure; and
many syntactic theories allow flat structures (e.g., Culicover \&
Jackendoff, 2006; Goldberg, 2006; Pollard \& Sag, 1994).  But if the
structure of the stimuli in the current experiments were required to be
binary-branching, both the descending (Figure~\ref{descstruc}) and flat
(Figure~\ref{flatstruc}) stimulus structures would be altered.  In the case
of the descending stimuli, this would have no influence on the predictions
of hierarchical feature-passing: N3 would still be more deeply-embedded and
more distant from the subject NP node than N2.  Changing the flat
structures to enforce binary branching, on the other hand, would alter the
predictions of hierarchical feature-passing.  The exact structural
alterations would depend on the specific theory, but because the first PP
modifies N1 and the second PP is independent of the first PP, the second PP
will attach above the first PP, by adjunction to the subject NP node (e.g.,
Carnie, 2005\IGNORE{pg 121}).  The result of this will be that the
hierarchical path for a feature from N3 to reach the top of the full
subject NP will be shorter than the path for a feature from N2, and thus a
hierarchical feature-passing approach built on binary-branching syntactic
structures would predict an even clearer interaction than the one described
for Experiment~1: The descending structures should show a larger mismatch
effect for N2 than for N3, while the ``flat'' structures should show a
larger mismatch effect for N3 than for N2.  Of course, Experiment~1 showed
no interaction at all: The difference between the N2 and N3 mismatch
effects was equivalent for the two different structures, with N2 yielding a
larger mismatch effect than N3.  Thus while we cannot evaluate hierarchical
feature-passing against every conceivable set of syntactic structures, its
key prediction of an interaction in Experiment~1 fails using either the
flat or the most likely binary-branched structures.

\IGNORE{\subsection{Reconsideration of Hierarchical Feature-Passing Results}}

Assuming, then, that the structures in the Experiment~1 and~2 stimuli were
constructed as intended, the experiments provide strong evidence that
hierarchical distance and hierarchical feature-passing are not relevant in
determining agreement error patterns.  This claim permits reconsideration
of some results in the literature, where the scope of planning account
might provide an alternative explanation.  As discussed above, Franck et
al.'s (2002) results are one such case, but three other agreement error
patterns have been explained at least in part with reference to
hierarchical mechanisms: First, sentential objects can produce mismatch
effects when they intervene between the subject NP and the verb\IGNORE{,
but these effects are smaller than from a local noun in a PP modifying the
subject NP} (Chanquoy \& Negro, 1996; Fayol, Largy, \& Lemaire, 1994;
Franck et al., 2006; Hartsuiker et al., 2001; see also Hemforth \&
Konieczny, 2003, for related data from German); second, the relative order
of the subject and verb can affect error rates (Franck et al., 2006;
Vigliocco \& Nicol, 1998); and, third, local nouns in PP modifiers produce
larger mismatch effects than local nouns in clausal modifiers (Bock \&
Cutting, 1992; Negro et al., 2005; Solomon \& Pearlmutter, 2004b).  How the
scope of planning account addresses these patterns is discussed below.

Hartsuiker et al.\ (2001) showed that local nouns in direct objects can
interfere with subject-verb agreement when they appear between the subject
noun and the verb, though they produce smaller mismatch effects than local
nouns in PP modifiers (cf.\ Franck et al., 2006; Hartsuiker et al.; and
references therein\IGNORE{ it would make sense to cite Ant\'on-M\'endez 96
directly here, but don't have it}; for more complex cross-linguistic
patterns involving object pronouns and clitics).  Examining Dutch,
Hartsuiker et al.\ presented participants with preambles
like~\exref{hetal}, consisting of a matrix clause followed by the start of
a subordinate clause and then an uninflected verb stem (\ext{WIN}) to be
used to complete the subordinate clause (Dutch subordinate clauses are
verb-final).  Both pairs produced mismatch effects, but larger effects
occurred for subordinate clauses containing a subject-modifying PP
(\ext{met de krans(en)} in~\exref{hetal-sm}) than for those containing a
direct object (\ext{de krans(en)} in~\exref{hetal-do}).  Hartsuiker et al.\
interpreted the presence of mismatch effects for both cases as support for
hierarchical feature-passing, and they interpreted the difference in
mismatch effect size as a consequence of the direct object noun's greater
hierarchical distance from the subject NP node: An interfering feature from
the direct object would have to pass up out of the direct object NP, to the
subordinate clause VP and/or the top of the subordinate clause itself, and
then to the subject NP; whereas in the PP-modifier case, an interfering
feature would only have to pass up out of the prepositional-object NP and
the PP itself (just as in corresponding English cases).

\begin{myexample}
\label{hetal}
\begin{examples}
    
    \item \label{hetal-sm} Karin zegt dat het meisje met de krans(en) 
    \hfill WIN \hspace{14em} \\
    Karen said that the girl with the garland(s) \hfill WIN \hspace{14em} 

    \item \label{hetal-do} Karin zegt dat het meisje de krans(en) 
    \hfill WIN \hspace{14em} \\
    Karen said that the girl the garland(s) \hfill WIN \hspace{14em} 
        
\end{examples}
\end{myexample}

Although a hierarchical feature-passing account can explain these results,
they do not provide direct evidence for such an account, and the scope of
planning explanation can also explain them.  In particular, if constituents
are generally planned in their utterance order (Bock, Loebell, \& Morey,
1992, provide some support for this view), the direct object NP
in~\exref{hetal-do}, like the corresponding NP in~\exref{hetal-sm}, will be
planned between the head noun of the subject NP and the verb, allowing
plurals in either case to create some interference in encoding the number
of the subject.  The difference between the two cases is likely created by
semantic integration: Most of the subject-modifying PPs in (the English
translations of) Hartsuiker et al.'s (2001) Experiment~1a and~1b stimuli
appear to be strongly integrated, which would yield relatively large
mismatch error effects.  In the direct object cases, we assume that verbs
can potentially create high levels of integration between their arguments
(e.g., Solomon \& Pearlmutter, 2004b, Exp.~5), but even assuming Hartsuiker
et al.'s uninflected target verbs did so, the link created by the target
verb in the direct object conditions would have been unavailable until
relatively late in the processing of the two NPs, compared to the
preposition in the subject-modifying PP case.  The result would be that the
subject and object would behave as if they were relatively unintegrated,
limiting the size of mismatch effects.  Thus the scope of planning account
can account for both the presence of mismatch effects in each case, as well
as the relatively larger effects in the subject-modifier
case~\exref{hetal-sm}.

An additional finding that the scope of planning account can address
without requiring a hierarchical component concerns word order variation.
In Franck et al.\ (2006; Exp.~1), participants were presented with an NP PP
subject phrase (in Italian) followed by an infinitival verb form
(e.g.,~\ref{franck06-exp1-stim}), and participants were required to
construct a sentence using those materials, inflecting the verb, that had
either a subject-verb~\exref{franck06-sv} or
verb-subject~\exref{franck06-vs} order.  More agreement errors were
observed in the subject-verb cases, where the local noun (\ext{ragazzi})
appeared between the head noun (\ext{vicino}) and the verb (\ext{viene}),
than in the verb-subject cases.  Franck et al.\ make several specific
assumptions about the syntactic representations and processes involved in
producing the subject-verb and verb-subject structures, in order to fit
them into their framework; but the critical property they use to
differentiate the two structures is linear precedence, and specifically
whether the local noun intervenes between the head noun and verb.\IGNORE{
argue that the low error rate in the VS ordering was due to the subject
remaining VP-internal (i.e., never raised to the Spec of the AgrS node),
and, thus agreement is only established by AGREE and is not subject to a
Spec-Head check which is "assumed to reinforce the morphological
realization of agreement" (p.  183).} Under a scope of planning account,
with hierarchical factors irrelevant, essentially this same linear order
factor would differentiate the two cases: Verb planning in verb-subject
cases would begin and could potentially be completed before the local noun
is planned, yielding few opportunities for interference; while in
subject-verb cases, the local noun will essentially always be planned
before the verb and will thus have ample opportunity to interfere with head
number tracking.  (An order of planning account can similarly explain
Franck et al.'s Experiment~3 result with mismatching local nouns in direct
objects, in which French speakers produced more errors in inflecting verbs
within object-verb-subject sequences than within object-subject-verb
sequences.  See also Hupet, Fayol, \& Schelstraete, 1998, for related
evidence from the French PP-inversion construction, which also creates an
object-verb-subject sequence.)

\begin{myexample}
\label{franck06-exp1}
\begin{examples}
    
    \item \label{franck06-exp1-stim} il vicino dei ragazzi 
    \hfill venire \hspace{19em} \\
    the neighbor of the boys \hfill to come \hspace{19em} 
    
    \item \label{franck06-sv} Il vicino dei ragazzi viene.
    \item \label{franck06-vs} Viene il vicino dei ragazzi.

\end{examples}
\end{myexample}

\IGNORE{
can't just complain about Exp2 -- same task as Exp1, but Exp2 has fewer 
misc errs (though also longer display times) -- probably best to make it 
an overall issue -- it's just a forced choice task for both expts.  other 
issue is mention in V\&N's GD that they ran another expt pair w/o adj 
presentation (preamble, then head N again, then S produces whole sentence 
or question) -- this expt produced very low err rates (duh) but still no 
decl vs interr diff\ldots --- try the following:}

Franck et al.\ (2006) point out that their Italian subject-verb versus
verb-subject construction results contrast with Vigliocco and Nicol's
(1998) finding that local nouns in English declaratives (subject-verb word
order; e.g., \ext{The helicopter for the flights is safe.})\ and
interrogatives (verb-subject; e.g., \ext{Is the helicopter for the flights
safe?})\ produce equal agreement error mismatch effects.  Franck et al.\
account for the difference in patterns with the idea that the English
interrogative (unlike the Italian verb-subject case) is created from the
declarative after agreement has been computed, so the difference in
produced word order has no effect.  A scope of planning account would not
predict this lack of an interaction with word order; but the Vigliocco and
Nicol experiments also used an altered version of the typical agreement
error elicitation task, which may have contributed to the lack of an
interaction: In both of Vigliocco and Nicol's experiments, participants
were presented with an adjective followed by a subject NP\@.  In all cases,
including fillers, the response was the subject NP and then the adjective,
with either \ext{is} or \ext{are} inserted in the appropriate position
(after or before the subject noun phrase).  Because declaratives and
interrogatives were produced in separate experiments, a given participant
always produced the verb in the same position, and the task may have
operated essentially as a forced choice procedure between the two possible
verb forms, with an additional memory component.  Thus their observed
interference effects may not have been the result of natural agreement
processing.  Further experiments will be needed to determine whether these
constructions present a problem for a scope of planning account.

\IGNORE{---other Franck et al.\ papers?  NO?}

\IGNORE{---explain Bock \& Cutting -- clause-boundedness}

One effect that the scope of planning account cannot easily explain is the
difference in mismatch effect size for local nouns embedded in phrases
versus clauses.  Bock and Cutting (1992, Exp.~1) compared PP modifier
preambles (e.g., \ext{The editor of the history book(s)}) with
corresponding length-matched clausal modifier preambles (e.g., \ext{The
editor who rejected the book(s)}) and showed that mismatch effects were
larger for the phrasal cases.  They suggested an explanation in terms of
clause boundedness: the idea that clauses are planned independently, and
elements within separate clauses cannot interfere with each other.  Solomon
and Pearlmutter (2004b, Exp.~5) replicated this result and showed that it
could not be explained by semantic integration; integration between the
head noun (e.g., \ext{editor}) and the local noun (\ext{book(s)}) did not
differ across structures in either the Bock and Cutting or the Solomon and
Pearlmutter stimuli.  Differences in linear distance to the head noun also
cannot explain the effect, as the phrasal and clausal modifier stimuli in
both experiments were matched on number of intervening syllables.

One possible account of the phrasal versus clausal modifier results is that
clause boundedness influences agreement processes independently of scope of
planning factors, essentially as Bock and Cutting (1992) described.
However, Franck et al.\ (2002) showed that the phrasal versus clausal
modifier effect could instead be explained by a hierarchical distance
account, because the verb phrase and potentially other structure needed to
instantiate the clausal modifier embeds the local noun more deeply than in
the phrasal modifier case.  Solomon and Pearlmutter (2004b) noted that
their own results were compatible with an independent effect of either
clause boundedness or hierarchical feature-passing, and recent work by Bock
and colleagues (Eberhard et al., 2005) has assumed a hierarchical
feature-passing approach in general, although they did not discuss clausal
modifiers in particular.  The current results suggest hierarchical
feature-passing is unlikely to be the source of errors in these
constructions (see below), but neither Experiment~1 nor~2 examined clausal
cases, so this leaves open the possibility that the mechanism underlying
agreement computation is hierarchical feature-passing, but hierarchical
distance depends only on syntactic nodes associated with clausal structure
(e.g., CP, IP or S, VP)\@.  This would be a substantially different version
of hierarchical feature-passing than what has previously been proposed, but
we know of no data that rule it out.\IGNORE{ \NOTE{I'm tempted to leave
this out, b/c we're just speculating at this point, and there's really
nothing to refer to yet\ldots} A third possible explanation for the phrasal
versus clausal modifier result is that the nature of the semantic
relationship established between two elements linked with a function word
(e.g., prepositions) and a content word (e.g., lexical verb) is
qualitatively different.  To date, all published studies examining the
interference effects of local nouns in clausal modifiers have used lexical
verbs in RCs, and the possible difference between prepositions and lexical
verbs may affects timing of planning of local nouns.}

\IGNORE{\subsection{Reconsidering Hierarchical Feature-Passing in the 
Marking \& Morphing Model}}

\IGNORE{---adjust M\&M model; w depends on planning distance, which depends
on planning time, based on linear order and semint (and maybe other stuff)}

In addition to alternative explanations for some results in the literature,
Experiments~1 and~2 also suggest a reconsideration of some aspects of
current agreement models.  In particular, because many of the results in
the literature have been connected to hierarchical feature-passing, current
models of agreement computation often incorporate a hierarchical component
(Eberhard et al., 2005; Vigliocco \& Hartsuiker, 2002; cf.\ Stevenson,
1994, in comprehension).  In Eberhard et al.'s (2005) marking and morphing
model, for example, as described above, the morphing process weights the
effect of each individual noun's number information on the overall subject
NP's number, based on that noun's hierarchical distance from the subject NP
node.  This weighting is critical for the model, because it enables it to
capture the controlling influence of the head noun over the local noun, and
thus the production of primarily grammatical agreement, when head noun
number and conceptual number diverge, as in ``distributive'' subject NPs
like \ext{The label on the bottles} (e.g., Bock et al., 2001; Eberhard,
1999; Vigliocco et al., 1996).  The current results would thus have two
main implications for this model: First, Experiment~1 suggests that
hierarchical distance might not be the appropriate determinant of weights,
as the model cannot account for the lack of an interaction with structure
in Experiment~1 if local noun weights are based on hierarchical distance.
Second, the scope account suggests the alternative of setting weights on
the basis of the relative timing of planning of number-bearing elements,
including the combination of linear distance to the head and semantic
integration seen in Experiment~2.  This change would allow the model to
handle both the Experiment~1 and~2 results, as well as Franck et al.'s
(2002) pattern and the effects attributed to hierarchical distance in
Hartsuiker et al.\ (2001), while still accounting for the cases
specifically modelled by Eberhard et al.\ (always NP PP
preambles).\footnote{Eberhard et al.\ note (p.~551) that their model
``takes no account of time, incrementality, or variations in syntactic
complexity apart from structural distance''; so our proposal might be seen
as remedying this while simultaneously removing any need for sensitivity to
structural distance.}

In fact, simply basing weights on linear distance to the head would be
sufficient to account for most of the results considered by Eberhard et
al.\ (2005).  Incorporating semantic integration into the weighting process
is probably necessary to account for the current Experiment~2 results as
well as the difference Hartsuiker et al.\ (2001) found between local nouns
in direct objects and in subject-modifying PPs; but the handling of
semantic integration in the marking and morphing model is also complicated
by two factors: First, there may be unidentified differences in integration
in the stimuli modeled by Eberhard et al.\ (see Solomon \& Pearlmutter's,
2004b, meta-analyses, for some discussion), which could obviously affect
the model's performance.  This can only be handled by gathering additional
semantic integration data and evaluating its effect on the model's fit to
the human error data.

The second complication is that Eberhard et al.\ (2005) did account for
Solomon and Pearlmutter's (2004b) overall semantic integration effect, but
they did so by way of the marking process: More integrated phrases were
assumed to mark the subject NP as a whole as more strongly plural, and,
specifically, they were treated as cases with ``ambiguous notional number,
such as subject phrases denoting masses\ldots\ collections, or distributions''
(Eberhard et al., p.~543).  This approach to semantic integration is in
principle a possibility, but it appears to run into some difficulty in its
application to various experiments: For example, in Solomon and
Pearlmutter's Experiment~1, the conceptual representation of the overall
preamble for the integrated versions (e.g., \ext{The drawing of the
flowers}) could be argued to be more like a mass because the local noun's
referent is incorporated into the head noun's; but it is still a single
entity, and the unintegrated versions (e.g., \ext{The drawing with the
flowers}) appear to refer more clearly to multiple entities, which would
suggest the opposite of what Solomon and Pearlmutter found.  Solomon and
Pearlmutter's Experiment~4 stimuli have similar properties, and it is
difficult to see how their Experiment~2 and~3 stimuli involve incorporation
or masses at all; intuitions suggest that both the integrated (e.g.,
\ext{The chauffeur for/of the actors}) and unintegrated stimuli (e.g.,
\ext{The chauffeur with the actors}) refer about equally to multiple
individuals.

The use of marking to handle semantic integration also runs into problems
with the current Experiment~2: Marking by definition applies to the subject
NP as a whole, and while we did not separately define an overall
integration measure, the early- and late-integrated cases are matched on
average integration of the three noun-noun relationships, so if marked
number depends on this, the model will fail to account for either of the
within-position comparisons between the early- and late-integrated
conditions (the mismatch effect from N2 was larger for early- than for
late-integrated cases, and the pattern reversed for N3 mismatch effects).
One could instead stipulate that marking's sensitivity to semantic
integration is based more heavily on the integration of N1 and N2, or
instead on the N1-N3 relationship, but either of these will make the wrong
prediction for the mismatch effect difference at the other position.  Using
the maximum semantic integration (or the minimum) of the pairs in the
subject NP fails as well, because the early- and late-integrated stimuli
are matched on these.

Given these issues, reliance on the marking process to handle the full
range of semantic integration effects thus seems problematic.  While
conceptual differences associated with semantic integration manipulations
can have correlated consequences for conceptual number (e.g., \ext{The
engine for the cars} is probably both conceptually more plural and more
semantically integrated than is \ext{The engine beside the cars}), the two
are separable.  Marking is certainly needed within the marking and morphing
model to handle cases where conceptual number of the subject NP diverges
from the number of any of its constituent lexical items (see, e.g.,
Eberhard et al.'s (2005, p.~536) discussion of the metonymic example
\ext{The hash browns at table nine is getting angry}); but semantic
integration effects appear to be independent of marking, and the current
work suggests that they can instead best be captured in Eberhard et al.'s
model by altering the weighting mechanism needed for morphing.

\IGNORE{---anything to say about Vig\&Hartsuiker? (iffy) Prob not\ldots}

\IGNORE{\subsection{Semantic Integration and Timing of Planning}}

\IGNORE{---more on semint}

\IGNORE{\TODO{the following para doesn't quite work --- not clear there's a
concern about the *design*; the issue is about the link between semint and
timing (and thus scope effects) --- so it's not a problem/concern, it's
stuff we don't yet know (and should perhaps go later in the GD); think we
want to claim that if semint affects timing, then it makes sense that both
linear order and semint should play into scope; the independence issue is
separate -- lack of a statistical interaction in HP2 doesn't speak directly
to this\ldots\ i.e., linear order and semint might just have independent
effects on timing; discussion of facts of non-interaction at end of para
also needs a little clarification} --- think this version works OK:}

The idea that both linear distance to the head and semantic integration
affect Eberhard et al.'s (2005) weighting parameter is a specific
implementation of the general claim that both of these properties
specifically affect timing of planning.  If language production proceeds at
least somewhat incrementally (e.g., Bock \& Levelt, 1994; Brown-Schimdt \&
Konopka, 2008; Brown-Schmidt \& Tanenhaus, 2006; Griffin, 2001), the linear
distance to the head manipulation straightforwardly varies timing of
planning of the local nouns in the current studies; however, the link
between timing and semantic integration is not as clear.  Solomon and
Pearlmutter (2004a, b) proposed that semantic integration effects on
agreement error rates resulted from more highly integrated elements being
planned with more overlap in timing, but while their results were
compatible with a timing account, they did not have direct evidence for
one.  Pearlmutter and Solomon (2007) argued for a timing account of
semantic integration based on exchange error patterns --- more integrated
phrases yielded more exchanges, suggesting that their elements were more
likely to be available simultaneously --- and the current findings provide
additional support for interpreting semantic integration effects as
reflecting timing, in that doing so allows for a unified explanation of the
results.  Placing both linear order and semantic integration on the same
temporal scale also suggests the possibility that they might interact, with
stronger or weaker effects of semantic integration at N2 versus at N3.
This turned out not to be the case, although the numeric pattern of effects
suggested a larger integration effect at N2 than at N3.  The lack of a
significant interaction may only indicate that the two factors are
independent influences on timing, but further research will be required to
establish that semantic integration affects timing of planning, and if it
does, how it interacts with other planning phenomena.

\IGNORE{

two additional important contrasts arising out of the current results:

1. Is agr computed over stable fully-constructed representations, or is it
computed during incremental processing, without all elements of subjNP
necessarily planned?  (or at least, which of these possible system
generates errors?)

2. mem-encoding vs mem-retrieval

\IGNORE{\subsection{Memory Encoding and Memory Retrieval}}

\IGNORE{---broad contrast between representation-driven agr effects and 
processing-driven effects?}

\NOTE{leaning toward dropping or reducing this next para\ldots\ pieces are
already mentioned elsewhere, and the set of suggested properties to be
considered I think was supposed to be more of a list of factors (which I
think is probably not worth trying to do here); Nicol 95 should be
mentioned, but there's a place earlier where we considered dropping it in
but didn't, and we could do so there} -- drop the whole thing:

A theory of agreement computation that relies on hierarchical relations to
explain mismatch effects require that the entire structure of the subject
NP is in place in order to compute agreement.  An account of agreement that
relies on memory encoding and a limited scope of planning can explain a
variety of effects described in the literature without requiring a complete
representation of the relevant structures in the utterance to be planned
prior to agreement computation.  Our account takes advantage of the
temporal and sequential nature of language production and is in accord with
other findings that suggest that planning in language production proceeds
at least somewhat incrementally (Brown-Schmidt \& Konopka, 2008;
Brown-Schmidt \& Tanenhaus, 2006; Griffin, 2001).\IGNORE{--- summary of
what should be in models} We suggest that models of agreement computation
should consider properties related to scope of planning, memory encoding,
and, possibly, memory retrieval.  This proposal suggests that models of
agreement should incorporate ways to quantify a local noun's planning
distance from the head noun in place of its syntactic distance (see Nicol,
1995 for additional discussion of temporal distance affecting agreement
error rates).  

}

\IGNORE{---hierarchy anyway, but limited by mem (pretty severely)?

\NOTE{THIS SUCKS COME BACK} While our results do not show support
for hierarchical distance influencing mismatch effects for phrasal
modifiers, it is still possible that the mechanism underlying agreement
computation is hierarchical feature-passing.  As mentioned above, when
semantic integration is controlled, the lower error rates found for RC
modifiers than PP modifiers may provide some evidence that this is the case
(Bock \& Cutting, 1992; Solomon \& Pearlmutter, 2004b).  Under a scope of
planning account, feature-passing would be limited by the amount of
structure that is planned and available to the system during agreement
computation.  If hierarchical feature-passing is the mechanism of agreement
computation, models would require a hierarchical component alongside a
planning distance component.  Under this assumption, hierarchical distance
would only have an effect on agreement error rates insomuch as equivalent
amounts of structure were planned that had different structural heights. }

\IGNORE{
\TODO{check if these have been handled:}

---other interps for semint effects?  \NOTE{this is sort of done w/what you
added about the marking and morphing stuff.  I don't know if we need much
more?}

---other interps for linear distance effects?  \NOTE{hmmm good question...
i can't really think of any}

---YUP; I think the above 2 issues are OK without more stuff}

\IGNORE{ 

following para was in 1st sub; replacing it w/Maureen's new 4 paras that 
follow\ldots

One other critical contrast arises in identifying the source(s) of errors
in agreement computation.  As noted briefly above, the scope of planning
account claims that errors arise during encoding of subject number into
memory during production: If a plural feature happens to be active in the
planning system around the time that a singular subject is being encoded,
the plural can interfere, eventually increasing the probability of
producing a plural instead of a singular verb (see also Nicol, 1995).  The
current experiments are compatible with these claims, and much of the
existing data in the agreement production literature (e.g., effects
attributed to hierarchical feature-passing, semantic integration effects,
and most or all of the effects on verb number captured by marking and
morphing and their interplay in the Eberhard et al.\ (2005) model) is
similarly compatible with the idea that errors occur when interference
arises in setting or tracking the number of the subject prior to whatever
agreement target (e.g., a verb) is eventually planned.  However, an
alternative source of agreement errors is interference at memory retrieval
(Badecker \& Kuminiak, 2007; Badecker \& Lewis, 2007): At the time of
planning an agreement target, the encoded source of number information for
that target must be retrieved from memory, and this process may be
susceptible to interference from other number-marked elements (e.g.,
intervening local nouns) that might be incorrectly retrieved instead of the
correct source.  This mechanism has difficulty explaining the patterns in
the current experiment, in that other things being equal, it predicts N3
should be more likely than N2 to trigger errors, because N3 has been
encountered more recently and is thus more accessible for retrieval (though
see Badecker \& Kuminiak (2007, p.~82) for a suggestion about how a
retrieval model might use ``planning chunks'' as cues); but it is
potentially compatible with many other patterns in the literature.
Furthermore, effects of phrase length (Bock \& Cutting, 1992) are more
naturally explained by retrieval interference than by encoding
interference.  This same contrast arises in agreement comprehension as
well, where initial work focused on what were essentially encoding-based
models (e.g., Nicol, Forster, \& Veres, 1997; Pearlmutter, 2000,
Pearlmutter, Garnsey, \& Bock, 1999), while several more recent
investigations of agreement have suggested that retrieval-based errors may
occur as well (e.g., H\"{a}ussler, 2006; Wagers, Lau, \& Phillips, 2009).
Lewis, Vasishth, and Van Dyke (2006) presented a model of memory in
comprehension in general which incorporates both encoding and retrieval
processes, and it may be that agreement production models will need to
incorporate both encoding and retrieval interference as potential sources
of agreement errors.

---

later version from Maureen, adjusted/expanded below

One other critical contrast arises in identifying the source(s) of errors
in agreement computation.  As noted briefly above, the scope of planning
account claims that errors arise during encoding of subject number into
memory during production: If a plural feature happens to be active in the
planning system around the time that a singular subject is being encoded,
the plural can interfere, eventually increasing the probability of
producing a plural instead of a singular verb (see also Nicol, 1995).  The
current experiments are compatible with these claims, and much of the
existing data in the agreement production literature (e.g., effects
attributed to hierarchical feature-passing, semantic integration effects,
and most or all of the effects on verb number captured by marking and
morphing and their interplay in the Eberhard et al.\ (2005) model) is
similarly compatible with the idea that errors occur when interference
arises in setting or tracking the number of the subject prior to whatever
agreement target (e.g., a verb) is eventually planned.  That being said,
our data do not neccessarily rule out a feature-passing mechanism, but
suggest that if feature-passing is the mechanism of agreement computation,
it must be constrained by the scope of planning.


Given how limited the effects of feature-passing would have to be, as our
data suggests that very little structure is planned and available to pass
features through, it may be beneficial to consider an alternative mechanism
of agreement computation that does not rely on feature-passing at all.

---

additional piece from Maureen, used with some adjustment below

However, an alternative source of agreement errors is interference at
memory retrieval (Badecker \& Kuminiak, 2007; Badecker \& Lewis, 2007).  At
the time of planning an agreement target (e.g., a verb), the encoded source
for that target must be retrieved from memory, and this process may be
susceptible to interference from other elements (e.g., intervening local
nouns) that might be incorrectly retrieved instead of the correct source.
Many retrieval models of comprehension incorporate similarity-based
retrieval interference to explain difficulties arising during the
processing of targets of long-distance dependencies (Gordon, Hendrick,
Johnson, \& Lee, 2006; Lewis, Vasishth, \& Van Dyke, 2006).  Under such an
account, retrieval interference could arise in at least two possible ways:
First, it could arise from retreiving the number marking of a local noun
rather than the number marking of the head noun, or it could arise from
mis-selecting a local noun as the subject of the sentence\footnote{All of
the critical items from 12 participants in Experiment~1 (288 sentences)
were examined to determine whether the predicate they produced clearly
referred to one of the local nouns rather than the head noun.  The majority
of predicates did not unambiguously specify which noun was selected as the
head; however, there were 6 unambiguous head mis-selection errors, but none
of them were categorized as agreement errors according to our coding
system.  Thus, we have no evidence that head mis-selection during retrieval
was the source of the error patterns observed in these experiments.}.  If
similarity-based retrieval interference were incorporated into a production
model, it would have difficulty explaining the pattern of results observed
in Experiment~2.  Under the definition of similarity proposed by these
accounts, the local nouns were equally dissimilar to the head noun: Neither
noun receives a grammatical subject flag and both local nouns refer to
inanimate objects.  Thus, the only other factor that should affect the
retrieval process is recency/locality.  In flat preambles like those in the
current experiments, with other things being equal, a retreival failure
account predicts that N3 should be more likely than N2 to trigger errors,
because N3 has been encountered more recently and is thus more accessible
for retrieval (though see Badecker \& Kuminiak (2007, p.~82) for a
suggestion about how a retrieval model might use ``planning chunks'' as
cues).  This is not to say that retrieval is not part of the agreement
production process.  Haskell and MacDonald (2005) provide evidence that
proximity to the verb plays a role in determining agreement for
disjunctions (e.g., \itt{the horse or the clocks}) such that the verb
overwhelmingly agrees with the noun that is proximial to the verb.  This
effect may be explained by the possibility that both nouns in a disjunction
receive a subject flag, and with all else being equal, the noun closest to
the verb provides the retreival cue for the verb as it was the most
recently encounted and most active noun with the correct grammatical
properties triggering agreement.  Furthermore, effects of phrase length
(Bock \& Cutting, 1992) are more naturally explained by retrieval
interference than by encoding interference as local nouns in longer PPs
cause more interference than local nouns in shorter PPs.

The same contrast between retrieval and encoding effects arises in
agreement comprehension as well, where initial work focused on what were
essentially encoding-based models (e.g., Nicol, Forster, \& Veres, 1997;
Pearlmutter, 2000, Pearlmutter, Garnsey, \& Bock, 1999), while several more
recent investigations of agreement have suggested that retrieval-based
errors may occur as well (e.g., H\"{a}ussler, 2006; Wagers, Lau, \&
Phillips, 2009).  The general comprehension model presented by Lewis,
Vasishth, and Van Dyke (2006) incorporates both memory encoding and
retrieval processes, and it may be that agreement production models will
need to incorporate both encoding and retrieval interference as potential
sources of agreement errors.


original mechanism paragraph:

widely-accepted Bock and Levelt (1994) model of production.  Their model
specifies how the subject of a sentence is identified and then linked to
the NP in the syntactic structure through functional assignment.  All that
would need to occur for agreement to be computed would be for the number
features of the lemma flagged as the subject to be encoded in working
memory and then this information could be retreived to build the structure
of the VP.

Under the scope of planning account assuming such a mechanism, interference
would arise when the number features of the local nouns' lemmas were
encoded in close temporal proximity to the head noun's lemma and the local
noun's number feature was associated with the subject NP during the
encoding process.  This hypothesized mechanism requires no feature-passing,
and takes advantage of the cascading, activation-based properties of the
Bock and Levelt (1994) model.  While it is not yet known if this mechanism
can explain all interference effects, it is an intriguing possibility that
warrants future research.

}


Beyond the specific consequences of our results for the existing literature
is the more general question about the source of agreement errors and the
mechanism of agreement computation.  Experiments~1 and~2 do not directly
rule out hierarchical feature-passing as the mechanism implementing
agreement; but they show that the evidence purportedly supporting it in the
literature is confounded and thus inconclusive, and we argued above that
with a scope of planning account, feature passing is not needed to explain
known results.  In addition, hierarchical feature-passing cannot be the
source of the error patterns in Experiments~1 and~2; the experiments
implicate factor(s) related to temporal planning distance from the head of
the subject NP, rather than syntactic distance from the subject NP\@.
Furthermore, if feature passing is the mechanism implementing agreement,
the current results severely limit its application and especially its
potential as a source of errors, because the explanation for the lack of an
influence of structure in Experiment~1 is that the structure associated
with the second PP --- through which errant features from N3 would have to
pass --- is not yet present when agreement properties of the subject are
computed.  At the same time, the structure corresponding to the first PP
must be present to account for the preponderance of N2 errors.  For a
feature-passing model, this means that feature passing (and consequent
feature-passing errors) can occur only within a very limited scope; namely,
within a phrase or so.  While feature passing might thus be responsible for
errors in the PP construction from a PP versus RC contrast (Bock \&
Cutting, 1992; Solomon \& Pearlmutter, 2004b), it would not be responsible
for errors from the RC construction (or, to the extent that they occur,
from full embedded clauses): We do not yet have data on the precise scope
that might be relevant here, but if structure has been planned for only a
single PP in the current experiments, it seems unlikely that structure for
an entire RC will have been planned in corresponding cases from Bock and
Cutting's and Solomon and Pearlmutter's experiments.  And if the point of a
feature-passing mechanism is to deliver subject number to the verb (e.g.,
Bock \& Levelt, 1994; Eberhard et al., 2005; Franck et al., 2002; Vigliocco
et al., 1996), some additional explanation for that part of the process
will be needed, given that the structure connected with the predicate will
not have been created when the subject's number is computed.  These are not
insurmountable problems, but they indicate that feature-passing accounts
cannot fully explain apparent effects of structure in the literature, and
that they are not complete even as accounts of the core phenomena or
operations for which they were developed.

As we noted above, the Experiment~1 and~2 results also point to an
alternative source of errors, which is the encoding of subject number into
memory: If a plural feature happens to be active in the planning system
around the time that a singular subject is being encoded, the plural can
interfere, eventually increasing the probability of producing a plural
instead of a singular verb (see also Nicol, 1995).  Along with the current
experiments, much of the existing data in the agreement production
literature (e.g., effects attributed to hierarchical feature-passing,
semantic integration effects, and most or all of the effects on verb number
captured by marking and morphing and their interplay in the Eberhard et
al.\ (2005) model) is compatible with the idea that errors occur when
interference arises in setting or tracking the number of the subject prior
to whatever agreement target (e.g., a verb) is eventually planned.  Indeed,
the Eberhard et al.\ model itself can be seen as primarily a model of the
memory encoding process for subject number.

However, if memory processes are responsible for agreement errors, then not
only encoding interference but also retrieval interference must be
considered, either as an alternative or in addition: At the time of
planning an agreement target (e.g., a verb), the encoded source for that
target must be retrieved from memory, and this process may be susceptible
to interference from other elements (e.g., intervening local nouns) that
might be incorrectly retrieved instead of the correct source (Badecker \&
Kuminiak, 2007; Badecker \& Lewis, 2007; Lewis \& Badecker, 2010).
Detailed models of memory retrieval processes in subject-verb agreement
production are only beginning to be discussed, but they have so far
primarily focused on two factors which have been proposed to play important
roles in retrieval models of long-distance dependency processing in
comprehension: recency (or decay) of activation and similarity-based
interference (Badecker \& Lewis; Gordon, Hendrick, Johnson, \& Lee, 2006;
Lewis \& Badecker; Lewis, Vasishth, \& Van Dyke, 2006).\footnote{A separate
question for retrieval-based models is whether the retrieval process
attempts to retrieve number information independently or instead attempts
to retrieve the head of the subject phrase, which in turn yields number
information; presumably the system might attempt to retrieve both, as this
might provide an alternative approach to combining number derived (in
Eberhard et al.'s (2005) terms) by marking and by morphing.  These
variations make different predictions about the extent to which
subject-verb agreement errors are correlated with head mis-selection errors
(cases in which the predicate is about a noun other than the head of the
subject; e.g., \ext{The baby on the blanket had tangled plaid fringe.}).
We attempted to evaluate this correlation by looking for head mis-selection
errors, using the continuations of all of the critical items from~12
participants in Experiment~1 (288 sentences), but the majority of
predicates did not unambiguously specify which noun was selected.  Of
the~42 cases that were unambiguous, only~4 were head mis-selection errors,
and none of them (nor any of the correct head-selection cases) was also an
agreement error case.  These data are obviously very limited, but they
certainly do not suggest that head mis-selection during retrieval was the
source of the error patterns in the current experiments.} However, the
effect of linear order in Experiment~2 (and in the flat conditions of
Experiment~1) seems to be a problem for both of these factors: Recency of
activation predicts that N3 should be more likely to be incorrectly
retrieved than N2, because N3 has been more recently produced and thus more
recently activated; but this is the reverse of the pattern in Experiment~2.
An alternative to recency of activation is frequency of activation (an
element retrieved multiple times during processing would be more likely to
interfere than one retrieved less often; cf.\ Lewis \& Badecker); but in
producing the flat conditions, N2 and N3 should be retrieved equally often,
predicting no difference in interference.  This also incorrectly predicts
the same interaction in Experiment~1 as hierarchical feature-passing,
because in the descending cases, unlike the flat cases, N2 should be
retrieved more frequently than N3 (N2 must be retrieved as a modifier of N1
and then again when it is modified by the PP containing N3); yet the
descending and flat cases yielded the same difference between N2 and N3
mismatch effects.\IGNORE{ Lewis and Badecker found that removing the decay
component from their original production model did not decrease model
performance, and suggested that all interference effects in subject-verb
agreement result from similarity-based interference.  Similarity-based
interference, however,} Similarity-based interference, similarly, does not
differentiate N2 and N3, as neither noun (or perhaps local NP) is tagged as
a grammatical subject (e.g., Bock \& Levelt, 1994), is in a subject-head
position (e.g., Lewis \& Badecker), or is marked with nominative case
(Badecker \& Kuminiak), and both nouns are inanimate and at the same
structural depth (but see Badecker \& Kuminiak (p.~82) for a suggestion
about how a retrieval model might use ``planning chunks'' as cues).

While memory retrieval thus appears to be insufficient on its own to
explain the full range of error patterns, it obviously could nevertheless
play a role in combination with memory encoding: Badecker and Kuminiak
(2007) argue for a retrieval-interference model of agreement based on an
interaction between case-marking and gender agreement in Slovak; and
Haskell and MacDonald (2005) show that for disjunctive subjects (e.g.,
\itt{the horse or the clocks}), in which the two nouns might be matched in
subject properties, the verb overwhelmingly agrees with the noun that is
proximal to the verb.\IGNORE{ This effect may be explained by the
possibility that both nouns in a disjunction receive a subject flag, and
with all else being equal, the noun closest to the verb provides the
retreival cue for the verb as it was the most recently encounted and most
active noun with the correct grammatical properties triggering agreement.}
Furthermore, effects of phrase length (Bock \& Cutting, 1992) are more
naturally explained by retrieval interference than by encoding
interference, as local nouns in longer PPs cause more interference than
local nouns in shorter PPs.

\IGNORE{this was the 2nd submission version of the preceding two 
paragraphs:

However, if memory processes are responsible for agreement errors, then not
only encoding interference but also retrieval interference must be
considered, either as an alternative or in addition: At the time of
planning an agreement target (e.g., a verb), the encoded source for that
target must be retrieved from memory, and this process may be susceptible
to interference from other elements (e.g., intervening local nouns) that
might be incorrectly retrieved instead of the correct source (Badecker \&
Kuminiak, 2007; Badecker \& Lewis, 2007).  Detailed models of memory
retrieval processes in subject-verb agreement production are only beginning
to be discussed, but they have so far focused on two factors which have
been proposed to play important roles in retrieval models of long-distance
dependency processing in comprehension: similarity-based interference and
recency of activation (Badecker \& Lewis; Gordon, Hendrick, Johnson, \&
Lee, 2006; Lewis, Vasishth, \& Van Dyke, 2006).\footnote{A separate
question for retrieval-based models is whether the retrieval process
attempts to retrieve number information independently or instead attempts
to retrieve the head of the subject phrase, which in turn yields number
information; presumably the system might attempt to retrieve both, as this
might provide an alternative approach to combining number derived (in
Eberhard et al.'s (2005) terms) by marking and by morphing.  These
variations make different predictions about the extent to which
subject-verb agreement errors are correlated with head mis-selection errors
(cases in which the predicate is about a noun other than the head of the
subject; e.g., \ext{The baby on the blanket had tangled plaid fringe.}).
We attempted to evaluate this correlation by looking for head mis-selection
errors, using the continuations of all of the critical items from~12
participants in Experiment~1 (288 sentences), but the majority of
predicates did not unambiguously specify which noun was selected.  Of
the~42 cases that were unambiguous, only~4 were head mis-selection errors,
and none of them (nor any of the correct head-selection cases) was also an
agreement error case.  These data are obviously very limited, but they
certainly do not suggest that head mis-selection during retrieval was the
source of the error patterns in the current experiments.} However, the
effect of linear order in Experiment~2 (and in the flat conditions of
Experiment~1) seems to be a problem for both of these factors:
Similarity-based interference does not differentiate N2 and N3, as neither
noun (or perhaps local NP) is tagged as a grammatical subject (e.g., Bock
\& Levelt, 1994), is in a subject-head position, or is marked with
nominative case (Badecker \& Kuminiak), and both nouns are inanimate and at
the same structural depth (but see Badecker \& Kuminiak (p.~82) for a
suggestion about how a retrieval model might use ``planning chunks'' as
cues).  Recency of activation predicts that N3 should be more likely to be
incorrectly retrieved than N2, because N3 has been more recently produced
and thus more recently activated; but this is the reverse of the pattern in
Experiment~2.  This indicates that memory retrieval is not sufficient on
its own to explain errors, but it obviously could play a role in
combination with memory encoding:  Badecker and Kuminiak argue for a
retrieval-interference model of agreement based on an interaction between
case-marking and gender agreement in Slovak; and Haskell and MacDonald
(2005) show that for disjunctive subjects (e.g., \itt{the horse or the
clocks}), in which the two nouns might be matched in subject properties,
the verb overwhelmingly agrees with the noun that is proximal to the
verb.\IGNORE{ This effect may be explained by the possibility that both
nouns in a disjunction receive a subject flag, and with all else being
equal, the noun closest to the verb provides the retreival cue for the verb
as it was the most recently encounted and most active noun with the correct
grammatical properties triggering agreement.} Furthermore, effects of
phrase length (Bock \& Cutting, 1992) are more naturally explained by
retrieval interference than by encoding interference, as local nouns in
longer PPs cause more interference than local nouns in shorter PPs.

}

The same contrast between retrieval and encoding effects arises in
agreement comprehension as well, where initial work focused on what were
essentially encoding-based models (e.g., Nicol, Forster, \& Veres, 1997;
Pearlmutter, 2000, Pearlmutter, Garnsey, \& Bock, 1999), while several more
recent investigations of agreement have suggested that retrieval-based
errors may also occur (e.g., H\"{a}ussler, 2006; Wagers, Lau, \& Phillips,
2009).  The general comprehension model presented by Lewis, Vasishth, and
Van Dyke (2006) incorporates both memory encoding and retrieval processes,
and it may be that agreement production models will need to incorporate
both encoding and retrieval interference as potential sources of agreement
errors.

But whether or not memory retrieval is needed in addition to encoding to
explain agreement errors, this still leaves open the question of the
mechanism of agreement computation, if there is no feature passing over
structure.  In fact, however, a memory-based system for encoding and
retrieving subject number may provide most of the needed machinery on its
own, with the rest provided by properties and processes independently
needed: In addition to encoding relevant agreement information in working
memory when the subject is planned and retrieving it when the predicate is
planned, an agreement system must determine the relevant information to be
encoded, which means identifying the correct subject phrase, identifying
the head of that phrase, and combining the relevant message-level and
lexical number information associated with these elements.  The combination
process is the focus of Eberhard et al.'s (2005) model, and we discussed
above how it could operate without any need for hierarchical
feature-passing; the mechanism for this part of the process might just be a
weighted combination (e.g., implemented in terms of activation).  The other
two aspects depend on the syntax and semantics to link message-level
elements to the syntactic subject phrase (usually an NP) and to the lexical
head for that phrase (and to link the subject phrase to its head), but they
are not specific to agreement processes and will be needed in any
production system that generates grammatically well-formed and discourse-
and semantically-appropriate sentences (e.g., Bock \& Levelt, 1994; Bock et
al., 1992), regardless of agreement.  Using them specifically for agreement
only requires stipulating that part of the information linked to the
subject phrase from the message, and part of the information activated in
the lexicon for the head, is information needed for agreement (at least
conceptual and lexical number, respectively, for English).  Similarly at
the time of retrieval, the retrieved number information must be associated
with the appropriate predicate phrase and then applied appropriately to the
relevant head; but identification of the predicate and its head are also
processes needed independently.  This is obviously barely a sketch of a
mechanism for agreement computation, and future research will have to
examine it in more detail; but it suggests how agreement might operate
without any need for hierarchical feature-passing, and it takes advantage
of the cascading, activation-based properties of the Bock and Levelt and
Eberhard et al.\ models.

\IGNORE{

f-p handled 3 processes: getting num info from head to phrase (w/in phrase)
--- provided by independent processes; moving num info between different
phrases when appropriate (subj to VP) --- provided by WM; and preventing
num info from moving between phrases when appropriate (mostly) --- handled 
by weighting based on scope/activation

}\IGNORE{---no hierarchical feat-passing, actually mem-encoding; but still
might need mem retrieval too (what effects?  -- maybe length?), and also
need conceptual (marking) aspect to handle\ldots \NOTE{length effect deals
with this a bit (but we don't discuss length effect anywhere except below)}

}\IGNORE{---more on mem-encoding?}

\IGNORE{\subsection{Comprehension in Production Tasks}}

One final potential limitation of these studies is that the
fragment-completion task used to elicit errors necessarily involves a
comprehension component, which is presumably not usually a part of the
production process.  However, to the extent there is an issue here, it
holds for nearly all agreement error elicitation studies to date: Almost
every study has used a version of one of two basic methods, either
presenting preambles first (auditorily or visually) and having speakers
remember then recite and complete them (e.g., Bock \& Cutting, 1992; Bock
\& Eberhard, 1993; Bock \& Miller, 1991; Eberhard, 1999; Fayol et al.,
1994; Franck et al., 2002; Haskell \& MacDonald, 2003), or else presenting
them (visually) and having speakers read them aloud and complete them
(e.g., the current experiments, Bock \& Eberhard, 1993; Solomon \&
Pearlmutter, 2004b; Vigliocco et al., 1996; Vigliocco \& Nicol, 1998).  In
either variant, speakers must comprehend the presented preamble in order to
use it during production.

On the other hand, the comprehension aspects of these tasks seem unlikely
to have much of an influence on the results: First, both versions of the
task have revealed clear influences of message-level representations (e.g.,
distributivity effects in Eberhard, 1999, and Hartsuiker, Kolk, \& Huinck,
1999; and semantic integration effects in both Experiment~2 and Solomon \&
Pearlmutter, 2004b), indicating that production is being at least partly
driven by its normal source (the message).  Second, although both tasks
require comprehension, its potential influence during production is likely
to be greater in the read-aloud version (in which comprehension overlaps
with production) than in the comprehend-then-repeat version; yet the result
patterns seen with the two different tasks generally show few differences
(e.g., Bock \& Cutting, 1992, vs.\ Solomon \& Pearlmutter's Exp.~5; Bock \&
Eberhard, 1993; Gillespie \& Pearlmutter, 2010).

Nevertheless, we cannot entirely rule out possible influences of
comprehension, and because the scope of planning account in particular
relies on timing of planning of elements of the subject noun phrase to
explain agreement errors, future work will have to examine whether these
tasks alter timing of availability and thus the extent to which different
factors are relevant.  One possibility would be to conduct more detailed
comparisons of the task variants that involve comprehension; an alternative
is to develop a paradigm requiring speakers to formulate their utterances
from the message level without any comprehension involved (see Haskell \&
MacDonald, 2005, for one possibility).

\IGNORE{ Maureen's original comprehension-task paragraph

One general limitation of these studies is the task used to elicit errors.
The task in these studies requires a comprehension component, which is not
usually a part of the normal production process.  To the extent that this
is an issue for this study, it is an issue in all studies of elicited
agreement errors: Studies to date have presented preambles auditorily or
visually prior to speakers repeating and completing them as full sentences
(e.g., Bock \& Miller, 1991; Bock \& Cutting, 1992; Franck et al., 2002;
Haskell \& MacDonald, 2003) or speakers read preambles off a computer
screen and then go on to form complete sentences (e.g., Bock \& Ebehard,
Solomon \& Pearlmutter, 2004b; Vigliocco, Butterworth, \& Garrett, 1996;
Vigliocco \& Nicol, 1998).  The scope of planning account relies on timing
of planning of elements of the subject noun phrase, but it is unclear
whether the comprehension component may alter the timing of planning
processes required for natural production.  Given that the agreement
elicitation task has shortcomings, it is clear that speakers in our and
Solomon and Pearlmutter's (2004b) experiments were sensitive to meaning
differences that arise at the message level, as evidenced by semantic
integration effects.  Thus, grammatical encoding was influenced by message
level representations using the fragment-completion task, which suggests
that the fragment completion task may be a suitable approximation of the
natural production process as both draw upon message level representations
to drive the production process.  In the future, a paradigm that requires
speakers to formulate their utterance from the message level should be used
to eliminate potential influences from comprehension (see Haskell \&
MacDonald, 2005).
}

\IGNORE{\subsection{Conclusion}}

Finally, the current results show that agreement studies can inform us
about the scope and units of planning in language production.  Much of the
research on scope of planning has focused on properties that affect
phonological encoding by measuring effects of semantically- and
phonologically-related distractors and syntactic complexity on speech onset
times (Allum \& Wheeldon, 2007; Martin, Crowther, Knight, Tamborello II, \&
Yang, in press; Martin, Miller, \& Vu, 2004; Smith \& Wheeldon, 1999, 2001;
Wheeldon \& Lahiri, 2002).  While these findings show that phonological
encoding is affected by properties specified at other levels, there is so
far little direct evidence of scope of planning effects at the grammatical
encoding level (cf.\ G\'{o}mez Gallo \& Jaeger, 2009, for some recent
suggestive evidence; and Watson, Breen, \& Gibson, 2006, for evidence of
grammatical effects on prosodic structure).\IGNORE{ Recent work by
G\'{o}mez Gallo and Jaeger (2009) has shown that the complexity of upcoming
arguments influences verb choice, and ultimately the syntactic structure of
the utterance, suggesting that speakers do some advanced planning of
elements within an utterance and that this planning affects the grammatical
encoding process.  --- I dropped this sentence b/c it wasn't clear what it
was getting us --- just the suggestion that there's advance planning isn't
so strong, and saying the advance planning affects encoding is pretty much
tautological.} The results of the current studies provide evidence that
planning proceeds at least somewhat incrementally, such that elements
(phrasal heads, at a minimum) are generally planned in the order in which
they are to be produced, multiple elements may be overlappingly activated
based on conceptual-level factors, and the advanced planning of elements
can influence grammatical encoding processes such as agreement computation.
A major advantage of this account of agreement error production is that it
is in accord with explanations of other types of speech errors.  For
example, exchange and other ordering errors are thought to occur when the
interacting elements are simultaneously active, and the wrong element is
selected for production (Garrett, 1975, 1980; Pearlmutter \& Solomon, 2007;
see, e.g., Dell, 1986; Dell, Burger, \& Svec, 1997, for corresponding
explanations of phoneme ordering errors).  Thus, the proposed scope of
planning account, which relies on the degree to which elements are
overlappingly planned, links the findings of agreement error production
studies to the rich tradition of research examining lexical and
phonological errors in spontaneous and experimentally elicited speech,
suggesting that different kinds of speech errors can be linked to the same
source.

\clearpage

\section[References]{\center References}

\begin{description}

    \item Allum, P.~H., \& Wheeldon, L.~R\@.  (2007).  Planning scope in
    spoken sentence production: The role of grammatical units.
    \itt{Journal of Experimental Psychology: Learning, Memory, and
    Cognition}, \itt{33}, 791--810.
    
    \item Badecker, W., \& Kuminiak, F\@.  (2007).  Morphology, agreement
    and working memory retrieval in sentence production: Evidence from
    gender and case in Slovak.  \itt{Journal of Memory and Language},
    \itt{56}, 65--85.
    
    \item Badecker, W., \& Lewis, R\@.  (2007, March).  \itt{A new theory
    and computational model of working memory in sentence production:
    Agreement errors as failures of cue-based retrieval}.  Paper presented
    at the 20th Annual CUNY Conference on Human Sentence Processing, La
    Jolla, CA.

%     \item Badecker, W., \& Lewis, R\@.  (2009).  \itt{Agreement and
%     sentence formulation: The role of working memory (retrievals) in
%     language production}.  Unpublished manuscript, University of Arizona
%     (Tuscon, AZ) and University of Michigan (Ann Arbor, MI).
% 
    \item Berent, I., Pinker, S., Tzelgov, J., Bibi, U., \& Goldfarb, L\@.
    (2005).  Computation of semantic number from morphological information.
    \itt{Journal of Memory and Language}, \itt{53}, 342--358.
    
    \item Bock, K., \& Cutting, J.~C\@.  (1992).  Regulating mental energy:
    Performance units in language production.  \itt{Journal of Memory and
    Language}, \itt{31}, 99--127.

    \item Bock, K., \& Eberhard, K.~M\@.  (1993).  Meaning, sound and
    syntax in English number agreement.  \itt{Language and Cognitive
    Processes}, \itt{8}, 57--99.

    \item Bock, K., Eberhard, K.~M., Cutting, J.~C., Meyer, A.~S., \&
    Schriefers, H\@.  (2001).  Some attractions of verb agreement.
    \itt{Cognitive Psychology}, \itt{43}, 83--128.

    \item Bock, K., \& Levelt, W\@.  (1994).  Language production:
    Grammatical encoding.  In M.~Gernsbacher (Ed.), \itt{Handbook of
    psycholinguistics} (pp.~945--984).  San Diego, CA: Academic Press.
    
    \item Bock, K., Loebell, H., \& Morey, R\@.  (1992).  From conceptual 
    roles to structural relations:  Bridging the syntactic cleft.  
    \itt{Psychological Review}, \itt{99}, 150--171.

    \item Bock, K., \& Miller, C.~A\@.  (1991).  Broken agreement.
    \itt{Cognitive Psychology}, \itt{23}, 45--93.

    \item Bock, K., Nicol, J., \& Cutting, J.~C\@.  (1999).  The ties that
    bind: Creating number agreement in speech.  \itt{Journal of Memory and
    Language}, \itt{40}, 330--346.
    
    \item Brown-Schmidt, S., \& Konopka, A., E\@.  (2008).  Little houses
    and casas peque\~{n}as: Message formulation and syntactic form in
    unscripted speech with speakers of English and Spanish.
    \itt{Cognition}, \itt{109}, 274--280.
    
    \item Brown-Schmidt, S., \& Tanenhaus, M., K.\@.  (2006).  Watching the
    eyes when talking about size: An investigation of message formulation
    and utterance planning.  \itt{Journal of Memory and Language},
    \itt{54}, 592--609.
    
    \item Carnie, A. (2005).  \itt{Syntax: A generative introduction}.
    Malden, MA: Blackwell.
    
    \item Chanquoy, L., \& Negro, I\@.  (1996).  Subject-verb agreement
    errors in written productions: A study of French children and adults.
    \itt{Journal of Psycholinguistic Research}, \itt{25}, 553--570.
   
    \item Chomsky, N\@.  (1965).  \itt{Aspects of the theory of syntax}.
    Cambridge, MA: MIT Press.
   
    \item Chomsky, N\@.  (1995).  \itt{The minimalist program}.  Cambridge,
    MA: MIT Press.

    \item Clark, H.~H\@.  (1973).  The language-as-fixed-effect fallacy: A
    critique of language statistics in psychological research.
    \itt{Journal of Verbal Learning and Verbal Behavior}, \itt{12},
    335--359.
    
    \item Cohen, J., \& Cohen, P\@.  (1983).  \itt{Applied multiple
    regression/correlation analysis for the behavioral sciences (2nd
    ed.)}.  Hillsdale, NJ: Lawrence Erlbaum.
    
    \item Cuetos, F., \& Mitchell, D.~C\@.  (1988).  Cross-linguistic
    differences in parsing: Restrictions on the use of the Late Closure
    strategy in Spanish.  \itt{Cognition}, \itt{30}, 73--105.
    
    \item Culicover, P.~W., \& Jackendoff, R\@.  (2006).  The simpler
    syntax hypothesis.  \itt{Trends in Cognitive Sciences}, \itt{10},
    413--418.
    
    \item Dell, G.~S\@.  (1986).  A spreading activation theory of
    retrieval in language production.  \itt{Psychological Review},
    \itt{93}, 283--321.
    
    \item Dell, G.~S., Burger, L.~K., \& Svec, W.~R\@. (1997).  Language 
    production and serial order:  A functional analysis and a model.  
    \itt{Psychological Review}, \itt{104}, 123--147.
    
    \item Eberhard, K.~M\@.  (1997).  The marked effect of number on
    subject-verb agreement.  \itt{Journal of Memory and Language},
    \itt{36}, 147--164.

    \item Eberhard, K.~M\@.  (1999).  The accessibility of conceptual
    number to the processes of subject-verb agreement in English.
    \itt{Journal of Memory and Language}, \itt{41}, 560--578.

    \item Eberhard, K.~M., Cutting, J.~C., \& Bock, K\@.  (2005).  Making
    syntax of sense: Number agreement in sentence production.
    \itt{Psychological Review}, \itt{112}, 531--559.

    \item Fayol, M., Largy, P., \& Lemaire, P\@.  (1994).  When cognitive
    overload enhances subject-verb agreement errors: A study in French
    written language.  \itt{Quarterly Journal of Experimental Psychology},
    \itt{47A}, 437--464.
    
    \item Franck, J., Lassi, G., Frauenfelder, U.~H., \& Rizzi, L\@.
    (2006).  Agreement and movement: A syntactic analysis of attraction.
    \itt{Cognition}, \itt{101}, 173--216.
    
    \item Franck, J., Vigliocco, G., \& Nicol, J\@.  (2002).  Subject-verb
    agreement errors in French and English: The role of syntactic
    hierarchy.  \itt{Language and Cognitive Processes}, \itt{17}, 371--404.
    
    \item Frazier, L., \& Clifton, C., Jr.  (1996).  \itt{Construal}.
    Cambridge, MA: MIT Press.
    
    \item Garrett, M.~F\@.  (1975).  The analysis of sentence production.
    In G.~Bower (Ed.), \itt{Psychology of learning and motivation} (Vol.~9,
    pp.~133--177).  New York: Academic Press.
    
    \item Garrett, M.~F\@.  (1980).  Levels of processing in sentence
    production.  In B.~Butterworth (Ed.), \itt{Language production}
    (Vol.~1, pp.~177--220).  London: Academic Press.
    
    \item Gibson, E., Pearlmutter, N.~J., Canseco-Gonzalez, E., \& Hickok,
    G\@.  (1996).  Recency preference in the human sentence processing
    mechanism.  \itt{Cognition}, \itt{59}, 23--59.
    
    \item Gillespie, M., \& Pearlmutter, N.~J\@.  (2010, March).
    \itt{Simultaneity of planning increases interference during
    subject-verb agreement production}.  Poster presented at the 23rd
    Annual CUNY Conference on Human Sentence Processing, New York, NY.
    
    \item Goldberg, A.~E\@.  (2006).  \itt{Constructions at work: The
    nature of generalization in language}.  Oxford: Oxford Univ.\ Press.
    
    \item G\'{o}mez Gallo, C., \& Jaeger, T.~F\@.  (2009, March).
    \itt{Early verb choice and fluency as evidence for moderately
    incremental or possibly limited parallel sentence production}.  Paper
    presented at the 22nd Annual CUNY Conference on Human Sentence
    Processing, Davis, CA.
    
    \item Gordon, P.~C., Hendrick, R., Johnson, M., \& Lee, Y\@.  (2006).
    Similarity-based interference during language comprehension: Evidence
    from eye tracking during reading.  \itt{Journal of Experimental
    Psychology: Learning, Memory, and Cognition}, \itt{32}, 1304--1321.
    
    \item Griffin, Z.~M\@.  (2001).  Gaze durations during speech reflect
    word selection and phonological encoding.  \itt{Cognition}, \itt{82},
    B1--B14.

    \item Hartsuiker, R.~J., Ant\'{o}n-M\'{e}ndez, I., \& van Zee, M\@.
    (2001).  Object attraction in subject-verb agreement construction.
    \itt{Journal of Memory and Language}, \itt{45}, 546--572.
    
    \item Hartsuiker, R.~J., Kolk, H.~H.~J., \& Huinck, W.~J\@.  (1999).
    Agrammatic production of subject-verb agreement: The effect of
    conceptual number.  \itt{Brain and Language}, \itt{69}, 119--160.
    
    \item Haskell, T. R\@., \& MacDonald, M. C\@.  (2003).  Conflicting
    cues and competition in subject-verb agreement.  \itt{Journal of Memory
    and Language}, \itt{48}, 760--778.
    
    \item Haskell, T. R\@., \& MacDonald, M. C\@.  (2005).  Constituent
    structure and linear order in language production: Evidence from
    subject-verb agreement.  \itt{Journal of Experimental Psychology:
    Learning, Memory, and Cognition}, \itt{31}, 891--904.
    
    \item H\"{a}ussler, J.  (2006).  Disrupted agreement checking in 
    sentence comprehension.  \itt{Proceedings of the Eleventh ESSLLI 
    Student Session}, 39--50.
    
    \item Hemforth, B., \& Konieczny, L. (2003).  Proximity in agreement
    errors.  \itt{Proceedings of the 25th Annual Conference of the
    Cognitive Science Society}, 557--562.
    
    \item Hupet, M., Fayol, M., \& Schelstraete, M\@.  (1998).  Effects of
    semantic variables on the subject-verb agreement processes in writing.
    \itt{British Journal of Psychology}, \itt{89}, 59--75.
    
    \item Kayne, R.~S\@.  (1984).  \itt{Connectedness and binary
    branching}.  Dordrecht: Foris.
    
    \item Lewis, R.~L., \& Badecker, W\@.  (2010, March).  \itt{Sentence
    production and the declarative and procedural components of short term
    memory}.  Paper presented at the 23rd Annual CUNY Conference on Human
    Sentence Processing, New York, NY.

    \item Lewis, R.~L., Vasishth, S., \& Van Dyke, J.~A\@.  (2006).
    Computational principles of working memory in sentence comprehension.
    \itt{Trends in Cognitive Sciences}, \itt{10}, 447--454.
    
    \item Martin, R.~C., Crowther, J.~E., Knight, M., Tamborello II, F.~P.,
    \& Yang, C\@.  (in press).  Planning in sentence production: Evidence
    for the phrase as a default planning scope.  \itt{Cognition}.

    \item Martin, R.~C., Miller, M., \& Vu, H\@.  (2004).  Lexical-semantic
    retention and speech production: Further evidence from normal and
    brain-damaged participants for a phrasal scope of planning.
    \itt{Cognitive Neuropsychology}, \itt{21}, 625--644.
    
    \item Negro, I., Chanquoy, L., Fayol, M., \& Louis-Sidney, M\@.
    (2005).  Subject-verb agreement in children and adults: Serial or
    hierarchical processing?  \itt{Journal of Psycholinguistic Research},
    \itt{34}, 233--258.
    
    \item Nicol, J.~L\@.  (1995).  Effects of clausal structure on
    subject-verb agreement errors.  \itt{Journal of Psycholinguistic
    Research}, \itt{24}, 507--516.
    
    \item Nicol, J.~L., Forster, K.~I., \& Veres, C\@.  (1997).
    Subject-verb agreement processes in comprehension.  \itt{Journal of
    Memory and Language}, \itt{36}, 569--587.
    
    \item Pearlmutter, N.~J\@.  (2000).  Linear versus hierarchical
    agreement feature processing in comprehension.  \itt{Journal of
    Psycholinguistic Research}, \itt{29}, 89--98.
    
    \item Pearlmutter, N.~J., Garnsey, S.~M., \& Bock, K\@.  (1999).
    Agreement processes in sentence comprehension.  \itt{Journal of Memory
    and Language}, \itt{41}, 427--456.
    
    \item Pearlmutter, N.~J., \& Gibson, E\@.  (2001).  Recency in verb
    phrase attachment.  \itt{Journal of Experimental Psychology: Learning,
    Memory, and Cognition}, \itt{27}, 574--590.
    
    \item Pearlmutter, N.~J., \& Solomon, E.~S\@.  (2007, March).
    \itt{Semantic integration and competition versus incrementality in
    planning complex noun phrases}.  Paper presented at the 20th Annual
    CUNY Conference on Human Sentence Processing, San Diego, CA.
    
    \item Pollard, C., \& Sag, I.~A\@.  (1994).  \itt{Head-driven phrase
    structure grammar}.  Chicago: Univ.\ of Chicago Press.

    \item Schneider, W\@.  (1988).  Micro Experimental Laboratory: An
    integrated system for IBM PC compatibles.  \itt{Behavior Research
    Methods, Instruments, \& Computers}, \itt{20}, 206--217.
    
    \item Sch\"{u}tze, C.~T., \& Gibson, E\@.  (1999).  Argumenthood and
    English prepositional phrase attachment.  \itt{Journal of Memory and
    Language}, \itt{40}, 409--431.

    \item Smith, M., \& Wheeldon, L\@.  (1999).  High level processing
    scope in spoken sentence production.  \itt{Cognition}, \itt{73},
    205--246.

    \item Smith, M., \& Wheeldon, L\@.  (2001).  Syntactic priming in
    spoken sentence production: An online study.  \itt{Cognition},
    \itt{78}, 123--164.

    \item Solomon, E.~S., \& Pearlmutter, N.~J\@.  (2004a, March).
    \itt{Semantic integration and hierarchical feature-passing in sentence
    production}.  Poster presented at the 17th Annual CUNY Conference on
    Human Sentence Processing, College Park, MD.
    
    \item Solomon, E.~S., \& Pearlmutter, N.~J\@.  (2004b).  Semantic
    integration and syntactic planning in language production.
    \itt{Cognitive Psychology}, \itt{49}, 1--46.
    
    \item Stevenson, S\@.  (1994).  \itt{A competitive attachment model for
    resolving syntactic ambiguities in natural language processing} (Tech.\
    Rep.\ No.\ 18).  New Brunswick, NJ: Rutgers Univ.\ Center for Cognitive
    Science.
    
    \item Vigliocco, G., Butterworth, B., \& Garrett, M.~F\@.  (1996).
    Subject-verb agreement in Spanish and English: Differences in the role
    of conceptual constraints.  \itt{Cognition}, \itt{61}, 261--298.

    \item Vigliocco, G., \& Hartsuiker, R.~J\@.  (2002).  The interplay of
    meaning, sound, and syntax in sentence production.  \itt{Psychological
    Bulletin}, \itt{128}, 442--472.
    
    \item Vigliocco, G., \& Nicol, J\@.  (1994).  \itt{The role of
    syntactic tree structure in the construction of subject verb
    agreement}.  Unpublished manuscript, University of Arizona, Tucson.

    \item Vigliocco, G., \& Nicol, J\@.  (1998).  Separating hierarchical
    relations and word order in language production: Is proximity concord
    syntactic or linear?  \itt{Cognition}, \itt{68}, B13--B29.
    
    \item Wagers, M.~W., Lau, E.~F., \& Phillips, C\@.  (2009).  Agreement
    attraction in comprehension: Representations and processes.
    \itt{Journal of Memory and Language}, \itt{61}, 206--237.
    
    \item Watson, D., Breen, M., \& Gibson, E\@.  (2006).  The role of
    syntactic obligatoriness in the production of intonational boundaries.
    \itt{Journal of Experimental Psychology: Learning, Memory, and
    Cognition}, \itt{32}, 1045--1056.
    
    \item Wheeldon, L.~R., \& Lahiri, A\@.  (2002).  The minimal unit of
    phonological encoding: Prosodic or lexical word.  \itt{Cognition},
    \itt{85}, B31--B41.
    
    
\end{description}

\clearpage

\section[Appendix~A]{\center Appendix~A: Experiment~1 Stimuli}

The purely singular versions of the Experiment~1 stimuli are shown
below.  Items~1--12 are the descending stimuli; items~13--24 are the
flat stimuli.  The other versions were created by making either the
the second noun or the third noun plural (but not both).

\begin{enumerate}

%descending

\item The bookcase with the ornate carving on the wooden shelf
    
\item The car with the silly sticker on the chrome bumper
    
\item The uniform with the silver star on the felt badge
    
\item The coat with the nylon tag on the folded cuff
    
\item The castle with the flaming torch on the stone moat
    
\item The magazine with the accurate illustration in the lengthy article
    
\item The purse with the pink button on the side pocket
    
\item The suit with the jagged rip in the wide lapel
    
\item The bracelet with the glass bead on the tiny clasp
    
\item The backpack with the plastic buckle on the leather strap
    
\item The watch with the sparkling jewel on the hour hand
    
\item The apartment with the full closet in the narrow hallway

%flat
    
\item The catalog for the department store with the ripped binding
    
\item The fax about the bankrupt company with the torn cover sheet
    
\item The safe for the pricy necklace with the combination lock
    
\item The ball for the rowdy game with the thick stripe
    
\item The keyboard for the modern computer with the chipped key
    
\item The cucumber for the fresh salad with the brownish spot
    
\item The tomato for the tasty sandwich with the nasty bruise
    
\item The postcard of the roller coaster with the foreign stamp
    
\item The diagram of the giant skyscraper with the tricky graph
    
\item The hose for the gorgeous garden with the replaceable nozzle
    
\item The tunnel through the steep mountain to the gold mine
    
\item The highway to the western suburb with the steel guardrail

\end{enumerate}

\clearpage

\section[Appendix~B]{\center Appendix~B: Experiment~2 Stimuli}

The purely singular early-integrated versions of the Experiment~2
stimuli are shown below.  The other versions were created by varying
the number of the noun in each PP and by varying the order of the two
PPs.

\begin{enumerate}

\item The book with the torn page by the red pen
    
\item The shirt with the snug sleeve under the cardboard box
    
\item The ring with the fake diamond near the blueberry muffin
    
\item The apple with the brown bruise beside the wicker basket
    
\item The lamp with the halogen bulb beside the antique vase
    
\item The drill with the titanium bit under the wool sweater 
    
\item The receipt with the blurry price near the dirty towel
    
\item The tree with the dead branch by the old building
    
\item The pizza with the yummy topping beside the broken toaster
    
\item The blanket with the soft seam behind the filing cabinet
    
\item The bowl with the noticeable crack under the flannel sheet
    
\item The bike with the bent spoke behind the rickety shed
    
\item The chair with the wobbly leg near the pruned shrub
    
\item The laptop with the loud speaker near the framed mirror
    
\item The staircase with the iron railing by the narrow hallway
    
\item The fork with the crooked prong by the fresh peach
    
\item The rose with the prickly thorn near the gold bracelet
    
\item The candle with the long wick beside the oak bookcase
    
\item The church with the tall steeple by the grassy park
    
\item The plane with the icy wing behind the massive truck
    
\item The skirt with the tattered hem behind the closet door
    
\item The printer with the ink cartridge by the ticking clock
    
\item The drawer with the wooden inlay under the tangled wire
    
\item The newspaper with the controversial headline beside the plastic bag
    
\item The purse with the full pocket near the remote control
    
\item The sign with the wooden post near the deep puddle
    
\item The glove with the tight finger under the gaudy necklace
    
\item The coat with the ripped cuff by the orange ball
    
\item The radio with the cracked knob beside the leather belt
    
\item The store with the vintage register near the crowded sidewalk
    
\item The train with the piercing whistle beside the peaceful lake
    
\item The shoe with the knotted lace behind the mossy log
    
\item The sink with the leaky faucet under the ceiling fan
    
\item The cake with the gooey filling near the electric blender
    
\item The boat with the nylon sail behind the granite monument
    
\item The oven with the hot burner beside the metal shelf
    
\item The plant with the yellowing leaf by the spiral notebook
    
\item The jacket with the faulty zipper near the sturdy desk
    
\item The razor with the rusty blade beside the purple brush
    
\item The chain with the tarnished link behind the oil tank

\end{enumerate}

\clearpage

\section[Author Note]{\center Author Note}

This research was supported by NIH Grant R01 DC05237 to the second author.
We would like to thank Addya Bhomick, Abigail Cinamon, Adrienne Davis, Amy
DiBattista, Jessica Eleman, Lauren Hovey, Sandra Lechner, Keith Levin,
Alyson Pospisil, Linda Raibert, Jacqueline Reindl, Kathrin Ritter, and
Courtney Tallarico for help in collecting data and carefully transcribing
and coding responses.  Portions of this work were presented at the 2008
CUNY Sentence Processing Conference (Chapel Hill, NC), the 2008 Psychonomic
Society Conference (Chicago, IL), and the 2009 CUNY Sentence Processing
Conference (Davis, CA)\@.

\end{document}
